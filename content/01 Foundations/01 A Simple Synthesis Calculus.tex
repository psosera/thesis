\todo{Building up $λ_{syn}$ starting with simple types.
  \begin{itemize}
    \item Starting with the simply-typed lambda calculus with unit, develop a example-driven synthesis system.
    \item Refine the system with proof search techniques.
    \item Give examples, walkthrough the process.
    \item Develop the metatheory.
  \end{itemize}%
}

To build a foundation for program synthesis with types, we start with the simply-typed lambda calculus, \stlcu{}.
We can think of program synthesis as search through the infinite space of programs refined by a given specification.
Therefore, from \stlcu{}, we build a generator for well-typed $λ$-terms and then integrate a notion of specification---here, input/output examples---into the system to create a program synthesis calculus.
Because \stlcu{} is the simplest typed-functional programming language we could consider, it serves as an excellent example of the process of converting type systems into synthesis systems.

\begin{figure}
  \begin{center}
    \begin{tabular}{>{$}l<{$} >{$}r<{$} >{$}l<{$} l}
      τ & \bnfdef & τ_1 → τ_2 \bnfalt 1 & Types\\
      e & \bnfdef & x \bnfalt e_1\;e_2 \bnfalt λx:τ.\;e \bnfalt () & Terms \\
      v & \bnfdef & λx:τ.\;e \bnfalt () & Values \\
      ℰ & \bnfdef & ◻ \bnfalt ℰ\;e \bnfalt v\;ℰ & Evaluation Contexts \\
      Γ & \bnfdef & · \bnfalt x:τ, Γ & Typing Contexts \\
    \end{tabular} \\[12pt]
    \hrule
    \begin{gather*}
      \fbox{\mbox{$Γ ⊢ e : τ$}} \qquad
        \inferrule[t-var]
          {x ∈ Γ}
          {Γ ⊢ x : τ} \qquad
        \inferrule[t-app]
          {Γ ⊢ e_1 : τ_1 → τ_2 \\ Γ ⊢ e_2 : τ_1}
          {Γ ⊢ e_1\;e_2 : τ_2} \\
        \inferrule[t-lam]
          {x:τ_1, Γ ⊢ e : τ_2}
          {Γ ⊢ λx:τ_1.\;e : τ_1 → τ_2} \qquad
        \inferrule[t-unit]
          { }
          {Γ ⊢ () : 1} \\
      \fbox{\mbox{$e ⟶ e'$}} \qquad
        \inferrule[eval-ctx]
          {e ⟶ e'}
          {ℰ[e] ⟶ e'} \qquad
        \inferrule[eval-app]
          {}
          {(λx:τ.\;e)\;v ⟶ [v/x]e}
    \end{gather*}
  \end{center}

\hrule
\caption{\stlcu{} definition}
\label{fig:stlc-unit-defn}
\end{figure}


\section{Generating \texorpdfstring{$λ$}{λ}-terms}

\autoref{fig:stlc-unit-defn} gives the syntax and semantics for \stlcu{} which contains variables, functions, and application.
\stlcu{} also features the $\mkwd{Unit}$ type (written $1$) as a base case for the types $τ$ of the language.
On top of the standard type system, we define a small-step, call-by-value operational semantics for the language using evaluation contexts $ℰ$~\todo{cite}.
\todo{Likely add more detail of the language even though it is pedantic---this is the first system presented in the thesis!}

To generate \stlcu{} terms, we can simply enumerate terms according to the grammar given in \autoref{fig:stlc-unit-defn}.
Clearly, there are an infinite number of terms, but we can make the enumeration deterministic by enumerating in order of term size.
By term size, we mean AST node count.  For example, the term $λx:1.\;y\;x$ has size 5 because the type $1$ has size one, the application $y\;x$ has size 3, and the lambda itself contributes one additional AST node.

\begin{figure}
  \begin{center}
    \begin{tikzpicture}
      \begin{semilogyaxis}[
        xlabel=Size (AST nodes),
        ylabel=Count,
        xmin=0,
        xmax=16,
        ymin=1,
        ymax=3000000,
        domain=1:15,
        samples=15,
        legend entries={All untyped, All typed,$\mkwd{Unit}$ type},
        legend pos=north west
      ]
        \addplot table[x index=0, y index=1] {data/lambda-terms-counts.data};
        \addplot table[x index=0, y index=2] {data/lambda-terms-counts.data};
        \addplot table[x index=0, y index=3] {data/lambda-terms-counts.data};
      \end{semilogyaxis}
    \end{tikzpicture}
  \end{center}

\hrule
\caption[Closed $λ$-term counts]{Number of closed untyped and typed terms and terms of type $\mkwd{Unit}$ in \stlcu{}}
\label{fig:lambda-terms-counts}
\end{figure}


However, simple enumeration is not very practical, especially when we have the type system at our disposal.
To see why, consider the number of closed \stlcu-terms at a given size, \ie, terms generated from the empty typing context.
We define the size of the term to be the number of nodes in its abstract syntax tree (AST) according to the grammar from \autoref{fig:stlc-unit-defn}.
\autoref{fig:lambda-terms-counts} gives the number of such untyped terms, well-typed terms, and well-typed terms at type $\mkwd{Unit}$.
\footnote{%
  Note that the data points in \autoref{fig:lambda-terms-counts} indicate the number of terms that are \emph{exactly} of the given size.
  This is the reason why, \eg, the number of terms of type $\mkwd{Unit}$ does not monotonically increase as the size increases.
}
For example:
\begin{itemize}
  \item $()\;(λx:1.\;x)$ is an syntactically valid, yet ill-typed term of size five.
  \item $(λx:1 → 1.\;x)\;(λx:1.\;x)$ is a well-typed term of size nine.
  \item $()$ is a well-typed term of type $\mkwd{Unit}$ and size one.
\end{itemize}

Noting that the $y$-axis scale of \autoref{fig:lambda-terms-counts} is at logscale, we can see that the number of syntactically valid, but not necessarily well-typed terms is staggering.
Even in this extremely limited language, there are 2,188,020 such terms at size 15.
In contrast, if we limit our consideration to only the well-typed terms, we save an order of magnitude of work; there are only 693,046 well-typed terms of size 15 in \stlcu.~\todo{cite the counting lambda terms work}

We could refine our term generation strategy by enumerating syntactically well-typed terms and then filtering out terms that fail to type check.
This has the benefit of ensuring that any terms that we keep around will be well-typed.
However, we still pay the time and space overhead of generating such terms and then type checking them, even if we end up throwing them away in the end.
This is undesirable in the presence of the steep combinatorial explosion of terms as their size increases.
Rather than type check after the fact, we should \emph{integrate} type checking into term enumeration so that we only ever consider well typed terms.

To arrive at such an algorithm, let us start with the type checking judgment presented in \autoref{fig:stlc-unit-defn}: $Γ ⊢ e : τ$.
When implementing a type checker based on this judgment, we note that $Γ$ and $e$ serve as inputs and $τ$ serves as either an input or output (a point that we revisit in detail in \autoref{sec:specifying-lambda-terms}).
To derive a type-aware term enumeration system from this judgment, let's simply flip the inputs and outputs: $Γ$ and $τ$ will be inputs and the output will be an $e$.

\begin{figure}
  \begin{center}
    \begin{gather*}
      \fbox{\mbox{$Γ ⊢ τ ⇝ e$}} \qquad
        \inferrule[gen-var]
          {x ∈ Γ}
          {Γ ⊢ τ ⇝ x} \qquad
        \inferrule[gen-app]
          {Γ ⊢ τ_1 → τ_2 ⇝ e_1 \\ Γ ⊢ τ_1 ⇝ e_2}
          {Γ ⊢ τ_2 ⇝ e_1\;e_2} \\
        \inferrule[gen-lam]
          {x:τ_1, Γ ⊢ τ_2 ⇝ e}
          {Γ ⊢ τ_1 → τ_2 ⇝ λx:τ_1.\;e} \qquad
        \inferrule[gen-unit]
          { }
          {Γ ⊢ 1 ⇝ ()}
    \end{gather*}
  \end{center}

\hrule
\caption{\stlcu{} typed term generation}
\label{fig:stlc-unit-gen}
\end{figure}


\autoref{fig:stlc-unit-gen} gives the rules for \stlcu{} well-typed term enumeration, written $Γ ⊢ τ ⇝ e$.
Derivations in this judgment correspond to generated terms.
For example, we generate the well-typed closed term $(λx:1 → 1.\;x)\;(λx:1.\;x)$ through the following derivation:
\[
\inferrule*
  {\inferrule*
    {\inferrule*
      {x:1 → 1 ∈ x:1 → 1}
      {x:1 → 1 ⊢ 1 → 1 ⇝ x}}
    {· ⊢ (1 → 1) → (1 → 1) ⇝ (λx:1 → 1.\;x)} \\
  \inferrule*
    {\inferrule*
      {x:1 ∈ x:1}
      {x:1 ⊢ 1 ⇝ x}}
    {· ⊢ (1 → 1) ⇝ (λx:1.\;x)}
  }
  {· ⊢ 1 → 1 ⇝ (λx:1 → 1.\;x)\;(λx:1.\;x)}
\]

Because the rules for well-typed term enumeration mirror the rules for type checking, the derivations are also identical!
This intuition leads to two natural properties of well-typed term enumeration:

\begin{theorem}[Soundness of \stlcu{} term enumeration]
\label{thm:soundness-stlcu-term-enumeration}
If $Γ ⊢ τ ⇝ e$ then $Γ ⊢ e : τ$.
\end{theorem}
\begin{proof}
Proof by induction on the derivation $Γ ⊢ τ ⇝ e$.
\end{proof}

\begin{theorem}[Completeness of \stlcu{} term enumeration]
\label{thm:completeness-stlcu-term-enumeration}
If $Γ ⊢ e : τ$ then $Γ ⊢ τ ⇝ e$.
\end{theorem}
\begin{proof}
Proof by induction on the derivation $Γ ⊢ e : τ$.
\end{proof}

\autoref{thm:soundness-stlcu-term-enumeration} states that enumerated terms are well-typed, and \autoref{thm:completeness-stlcu-term-enumeration} states that we are able to enumerate the well-typed terms.
By combining both theorems, we know that well-typed term enumeration produces \emph{exactly} the set of well-typed terms, \ie, the ``All Typed'' dataset from \autoref{fig:lambda-terms-counts}.

\section{Specifying \texorpdfstring{$λ$}{λ}-terms}
\label{sec:specifying-lambda-terms}



\begin{figure}
  \begin{center}
    \begin{tabular}{>{$}l<{$} >{$}r<{$} >{$}l<{$} l}
      τ  & \bnfdef & τ_1 → τ_2 \bnfalt 1 & Types\\
      e  & \bnfdef & x \bnfalt e_1\;e_2 \bnfalt λx:τ.\;e \bnfalt () & Terms \\
      v  & \bnfdef & λx:τ.\;e \bnfalt () & Values \\
      ℰ  & \bnfdef & ◻ \bnfalt ℰ\;e \bnfalt v\;ℰ & Evaluation Contexts \\
      Γ  & \bnfdef & · \bnfalt x:τ, Γ & Typing Contexts \\
      \\
      E  & \bnfdef & x \bnfalt E\;I & Elimination Terms \\
      I  & \bnfdef & E \bnfalt λx:τ.\;I \bnfalt () & Introduction Terms \\
      σ  & \bnfdef & · \bnfalt [v/x]σ & Evaluation Environments \\
      χ  & \bnfdef & () \bnfalt pf & Example Values \\
      pf & \bnfdef & \many{v_i ⇒ χ_i}{i < m} & Partial Functions \\
      Χ  & \bnfdef & · \bnfalt σ ↦ χ, Χ &  Example Worlds
    \end{tabular} \\[12pt]
    \hrule
    \begin{gather*}
      \fbox{$Γ ⊢ e : τ$}\;\fbox{$e⟶ e'$} \qquad
        \text{(Same as \stlcu{})} \\
      \fbox{$Γ ⊢ E ⇒ τ$} \qquad
        \inferrule[t-Evar]
          {x:τ ∈ Γ}
          {Γ ⊢ x ⇒ τ} \qquad
        \inferrule[t-Eapp]
          {Γ ⊢ E ⇒ τ_1 → τ_2 \\ Γ ⊢ I ⇐ τ_1}
          {Γ ⊢ E\;I ⇒ τ_2} \\
      \fbox{$Γ ⊢ I ⇐ τ$} \qquad
        \inferrule[t-Ilam]
          {x:τ_1, Γ ⊢ I ⇐ τ_2}
          {Γ ⊢ λx:τ_1.\;I ⇐ τ_1 → τ_2} \qquad
        \inferrule[t-Iunit]
          { }
          {Γ ⊢ () ⇐ 1} \\
      \fbox{$Γ ⊢ χ : τ$} \qquad
        \inferrule[t-ex-unit]
          { }
          {Γ ⊢ () : \mkwd{Unit}} \qquad
        \inferrule[t-ex-pf]
          {\many{Γ ⊢ v_i : τ_1}{i < m} \\ \many{Γ ⊢ χ_i : τ_2}{i < m}}
          {Γ ⊢ \many{v_i ⇒ χ_i}{i < m} : τ_1 ⇒ τ_2} \\
      \fbox{$Γ ⊢ σ$} \qquad
        \inferrule[t-env-empty]
          { }
          {Γ ⊢ ·} \qquad
        \inferrule[t-env-cons]
          {x:τ ∈ Γ \\ ⊢ v : τ \\ Γ ⊢ σ}
          {Γ ⊢ [v/x]σ} \\
      \fbox{$Γ ⊢ Χ : τ$} \qquad
        \inferrule[t-ex-empty]
          { }
          {Γ ⊢ · : τ} \qquad
        \inferrule[t-ex-cons]
          {Γ ⊢ σ \\ Γ ⊢ χ : τ \\ Γ ⊢ Χ : τ}
          {Γ ⊢ σ ↦ χ, Χ : τ}
    \end{gather*}
  \end{center}

  \hrule
  \caption{\lsynu{} syntax and typechecking}
  \label{fig:lsyn-unit-defn}
\end{figure}


\begin{figure}
  \begin{center}
    \begin{gather*}
      \fbox{$Γ ⊢ τ ⇝ E$} \qquad
        \inferrule[eguess-var]
          {x:τ ∈ Γ}
          {Γ ⊢ τ ⇝ x} \qquad
        \inferrule[eguess-app]
          {Γ ⊢ E ⇝ τ_1 → τ_2 \\ Γ ⊢ τ_1 ▷ · ⇝ I}
          {Γ ⊢ τ_2 ⇝ E\;I} \\
      \fbox{$Γ ⊢ τ ▷ Χ ⇝ I$} \qquad
        \inferrule[irefine-arr]
          {X = σ_1 ↦ pf_1, …, σ_n ↦ pf_n \\\\
          Χ' = \mfun{apply}(x, σ_1, pf_1) \concat … \concat \mfun{apply}(x, σ_n, pf_n) \\\\
          x:τ_1, Γ ⊢ τ_2 ▷ Χ' ⇝ I}
          {Γ ⊢ τ_1 → τ_2 ▷ Χ ⇝ λx:τ_1.\;I} \\
        \inferrule[irefine-unit]
          {Χ = σ_1 ↦ (), …, σ_n ↦ ()}
          {Γ ⊢ 1 ▷ Χ ⇝ ()} \qquad
        \inferrule[irefine-guess]
          {Γ ⊢ τ ⇝ E \\ E ⊢ Χ}
          {Γ ⊢ τ ▷ Χ ⇝ E} \\
      \fbox{$I ⊢ Χ$} \qquad
        \inferrule[satisfies]
          {∀σ ↦ χ ∈ Χ.\;σ(I) ⟶^* v ∧ v ≃ χ}
          {I ⊢ Χ} \\
      \fbox{$v ≃ χ$} \qquad
        \inferrule[eq-unit]
          { }
          {() ≃ ()} \qquad
        \inferrule[eq-lam-pf]
          {∀i ∈ 1, …, m.\; (λx:τ.\;e)\;v_i ⟶^* v ∧ v ≃ χ_i}
          {λx:τ.\;e ≃ \many{v_i ⇒ χ_i}{i < m}} \\
      \mfun{apply}(x, σ, \many{v_i ⇒ χ_i}{i < m}) = [v_i/x]σ ↦ χ_i, …, [v_m/x]σ ↦ χ_m
    \end{gather*}
  \end{center}

  \hrule
  \caption{\lsynu{} synthesis and equivalence}
  \label{fig:lsyn-unit-synthesis}
\end{figure}

