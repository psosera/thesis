\todo{Building up $λ_{syn}$ starting with simple types.
  \begin{itemize}
    \item Starting with the simply-typed lambda calculus with unit, develop a example-driven synthesis system.
    \item Refine the system with proof search techniques.
    \item Give examples, walkthrough the process.
    \item Develop the metatheory.
  \end{itemize}%
}

To build a foundation for program synthesis with types, we start with the simply-typed lambda calculus, \stlcu{}.
We can think of program synthesis as search through the infinite space of programs refined by a given specification.
Therefore, from \stlcu{}, we build a generator for well-typed $λ$-terms and then integrate a notion of specification---here, input/output examples---into the system to create a program synthesis calculus.
Because \stlcu{} is the simplest typed-functional programming language we could consider, it serves as an excellent example of the process of converting type systems into synthesis systems.

\begin{figure}
  \begin{center}
    \begin{tabular}{>{$}l<{$} >{$}r<{$} >{$}l<{$} l}
      τ & \bnfdef & τ_1 → τ_2 \bnfalt 1 & Types\\
      e & \bnfdef & x \bnfalt e_1\;e_2 \bnfalt λx:τ.\;e \bnfalt () & Terms \\
      v & \bnfdef & λx:τ.\;e \bnfalt () & Values \\
      ℰ & \bnfdef & ◻ \bnfalt ℰ\;e \bnfalt v\;ℰ & Evaluation Contexts
    \end{tabular} \\[12pt]
    \hrule
    \begin{gather*}
      \fbox{\mbox{$Γ ⊢ e : τ$}} \qquad
        \inferrule[t-var]
          {x ∈ Γ}
          {Γ ⊢ x : τ} \qquad
        \inferrule[t-app]
          {Γ ⊢ e_1 : τ_1 → τ_2 \\ Γ ⊢ e_2 : τ_1}
          {Γ ⊢ e_1\;e_2 : τ_2} \\
        \inferrule[t-lam]
          {x:τ_1, Γ ⊢ e : τ_2}
          {Γ ⊢ λx:τ_1.\;e : τ_1 → τ_2} \qquad
        \inferrule[t-unit]
          { }
          {Γ ⊢ () : 1} \\
      \fbox{\mbox{$e ⟶ e'$}} \qquad
        \inferrule[eval-ctx]
          {e ⟶ e'}
          {ℰ[e] ⟶ e'} \qquad
        \inferrule[eval-app]
          {}
          {(λx:τ.\;e)\;v ⟶ [v/x]e}
    \end{gather*}
  \end{center}

\hrule
\caption{\stlcu{} definition}
\label{fig:stlc-unit-defn}
\end{figure}


\section{Generating \texorpdfstring{$λ$}{λ}-terms}

\autoref{fig:stlc-unit-defn} gives the syntax and semantics for \stlcu{} which contains variables, functions, and application.
\stlcu{} also features the $\mkwd{Unit}$ type (written $1$) as a base case for the types $τ$ of the language.
On top of the standard type system, we define a small-step, call-by-value operational semantics for the language using evaluation contexts $ℰ$~\todo{cite}.
\todo{Likely add more detail of the language even though it is pedantic---this is the first system presented in the thesis!}

To generate \stlcu{} terms, we can simply enumerate terms according to the grammar given in \autoref{fig:stlc-unit-defn}.
Clearly, there are an infinite number of terms, but we can make the enumeration deterministic by enumerating in order of term size.
By term size, we mean AST node count.  For example, the term $λx{:}1.\;y\;x$ has size 5 because the type $1$ has size one, the application $y\;x$ has size 3, and the lambda itself contributes one additional AST node.

\begin{figure}
  \begin{center}
    \begin{tikzpicture}
      \begin{semilogyaxis}[
        xlabel=Size (AST nodes),
        ylabel=Count,
        xmin=0,
        xmax=16,
        ymin=1,
        ymax=3000000,
        domain=1:15,
        samples=15,
        legend entries={All untyped, All typed,$\mkwd{Unit}$ type},
        legend pos=north west
      ]
        \addplot table[x index=0, y index=1] {data/lambda-terms-counts.data};
        \addplot table[x index=0, y index=2] {data/lambda-terms-counts.data};
        \addplot table[x index=0, y index=3] {data/lambda-terms-counts.data};
      \end{semilogyaxis}
    \end{tikzpicture}
  \end{center}

\hrule
\caption[Closed $λ$-term counts]{Number of closed untyped and typed terms and terms of type $\mkwd{Unit}$ in \stlcu{}}
\label{fig:lambda-terms-counts}
\end{figure}


However, simple enumeration is not very practical, especially when we have the type system at our disposal.
To see why, consider the number of closed \stlcu-terms at a given size, \ie, terms generated from the empty typing context.
We define the size of the term to be the number of nodes in its abstract syntax tree (AST) according to the grammar from \autoref{fig:stlc-unit-defn}.
\autoref{fig:lambda-terms-counts} gives the number of such untyped terms, well-typed terms, and well-typed terms at type $\mkwd{Unit}$.
\footnote{%
  Note that the data points in \autoref{fig:lambda-terms-counts} indicate the number of terms that are \emph{exactly} of the given size.
  This is the reason why, \eg, the number of terms of type $\mkwd{Unit}$ does not monotonically increase as the size increases.
}
For example:
\begin{itemize}
  \item $()\;(λx{:}1.\;x)$ is an syntactically valid, yet ill-typed term of size five.
  \item $(λx{:}1 → 1.\;x)\;(λx{:}1.\;x)$ is a well-typed term of size nine.
  \item $()$ is a well-typed term of type $\mkwd{Unit}$ and size one.
\end{itemize}

Noting that the $y$-axis scale of \autoref{fig:lambda-terms-counts} is at logscale, we can see that the number of syntactically valid, but not necessarily well-typed terms is staggering.
Even in this extremely limited language, there are 2,188,020 such terms at size 15.
In contrast, if we limit our consideration to only the well-typed terms, we save an order of magnitude of work; there are only 693,046 well-typed terms of size 15 in \stlcu.~\todo{cite the counting lambda terms work}

We could refine our term generation strategy by enumerating syntactically well-typed terms and then filtering out terms that fail to type check.
This has the benefit of ensuring that any terms that we keep around will be well-typed.
However, we still pay the time and space overhead of generating such terms and then type checking them, even if we end up throwing them away in the end.
This is undesirable in the presence of the steep combinatorial explosion of terms as their size increases.
Rather than type check after the fact, we should \emph{integrate} type checking into term enumeration so that we only ever consider well typed terms.

To arrive at such an algorithm, let us start with the type checking judgment presented in \autoref{fig:stlc-unit-defn}: $Γ ⊢ e : τ$.
When implementing a type checker based on this judgment, we note that $Γ$ and $e$ serve as inputs and $τ$ serves as either an input or output (a point that we revisit in detail in \autoref{sec:specifying-lambda-terms}).
To derive a type-aware term enumeration system from this judgment, let's simply flip the inputs and outputs: $Γ$ and $τ$ will be inputs and the output will be an $e$.

\begin{figure}
  \begin{center}
    \begin{gather*}
      \fbox{\mbox{$Γ ⊢ τ ⇝ e$}} \qquad
        \inferrule[gen-var]
          {x ∈ Γ}
          {Γ ⊢ τ ⇝ x} \qquad
        \inferrule[gen-app]
          {Γ ⊢ τ_1 → τ_2 ⇝ e_1 \\ Γ ⊢ τ_1 ⇝ e_2}
          {Γ ⊢ τ_2 ⇝ e_1\;e_2} \\
        \inferrule[gen-lam]
          {x:τ_1, Γ ⊢ τ_2 ⇝ e}
          {Γ ⊢ τ_1 → τ_2 ⇝ λx:τ_1.\;e} \qquad
        \inferrule[gen-unit]
          { }
          {Γ ⊢ 1 ⇝ ()}
    \end{gather*}
  \end{center}

\hrule
\caption{\stlcu{} typed term generation}
\label{fig:stlc-unit-gen}
\end{figure}


\autoref{fig:stlc-unit-gen} gives the rules for \stlcu{} well-typed term enumeration, written $Γ ⊢ τ ⇝ e$.
Derivations in this judgment correspond to generated terms.
For example, we generate the well-typed closed term $(λx{:}1 → 1.\;x)\;(λx{:}1.\;x)$ through the following derivation:
\[
\inferrule*
  {\inferrule*
    {\inferrule*
      {x{:}1 → 1 ∈ x{:}1 → 1}
      {x{:}1 → 1 ⊢ 1 → 1 ⇝ x}}
    {· ⊢ (1 → 1) → (1 → 1) ⇝ (λx{:}1 → 1.\;x)} \\
  \inferrule*
    {\inferrule*
      {x{:}1 ∈ x{:}1}
      {x{:}1 ⊢ 1 ⇝ x}}
    {· ⊢ (1 → 1) ⇝ (λx{:}1.\;x)}
  }
  {· ⊢ 1 → 1 ⇝ (λx{:}1 → 1.\;x)\;(λx{:}1.\;x)}
\]

Because the rules for well-typed term enumeration mirror the rules for type checking, the derivations are also identical!
This intuition leads to two natural properties of well-typed term enumeration:

\begin{theorem}[Soundness of \stlcu{} term enumeration]
\label{thm:soundness-stlcu-term-enumeration}
If $Γ ⊢ τ ⇝ e$ then $Γ ⊢ e : τ$.
\end{theorem}
\begin{proof}
Proof by induction on the derivation $Γ ⊢ τ ⇝ e$.
\end{proof}

\begin{theorem}[Completeness of \stlcu{} term enumeration]
\label{thm:completeness-stlcu-term-enumeration}
If $Γ ⊢ e : τ$ then $Γ ⊢ τ ⇝ e$.
\end{theorem}
\begin{proof}
Proof by induction on the derivation $Γ ⊢ e : τ$.
\end{proof}

\autoref{thm:soundness-stlcu-term-enumeration} states that enumerated terms are well-typed, and \autoref{thm:completeness-stlcu-term-enumeration} states that we are able to enumerate the well-typed terms.
By combining both theorems, we know that well-typed term enumeration produces \emph{exactly} the set of well-typed terms, \ie, the ``All Typed'' dataset from \autoref{fig:lambda-terms-counts}.

\subsection{Enumerating Normal Forms}
\label{subsec:enumerating-normal-forms}

By restricting term generation to well-typed terms, we substantially reduce the search space of programs.
However there are still many redundant terms that we could generate.
In particular, the following terms are redundant
\begin{itemize}
  \item $(λx{:}1.\;x)\;()$
  \item $(λx{:}1 → 1.\;x)\;(λy{:}1.\;y)$
  \item $λx{:}1.\;(λy{:}1.\;y)\;()$.
\end{itemize}
In the first two cases, the terms evaluate, according to our call-by-value semantics, to the values $()$ and $λy{:}1.\;y$, respectively.
In the third case, while the term is already a value, the body of the lambda reduces to produce the equivalent term $λx{:}1.\;()$.
In general, any term that is not in \emph{$β$-normal form}---a term where no $β$-reductions apply----is redundant with the term that is $β$-equivalent to it.\footnote{%
  This is true of \stlcu{} because it is strongly normalizing.
  In the presence of non-termination, terms with infinite loops do not have a corresponding normal form.
}

To restrict enumeration to the well-typed normal forms of \stlcu{}, we restrict terms so that they cannot contain pending $β$-reductions.
In \stlcu{}, the only $β$-reduction available is function application which applies whenever we have a term of the form $(λx{:}τ.\;e)\;v$.
We avoid this situation by splitting expressions into two sets of sub-terms.
\begin{align*}
  E &\bnfdef x \bnfalt E\;I \\
  I &\bnfdef E \bnfalt λx{:}τ.\;I \bnfalt ()
\end{align*}
$E$-forms include variables and \emph{elimination} forms, and $I$-forms include \emph{introduction} forms.
For example, lambdas introduce values of arrow type and function application eliminates these values.
In contrast, $\mkwd{Unit}$ values are introduced with $()$ and have no corresponding elimination forms.

With this syntax, $I$ and $E$ forms are in normal form by construction!
To see this, note that function application requires that the head of a function application is an $E$, and the syntax of $E$s bottom outs at variables.
Therefore, all applications will be of the form $x I_1, …, I_m$ where a variable is always at the head of the application.
Arguments, in contrast, are allowed to be any $I$-term which includes $E$s because $E$s are included in the definition of $I$.

This technique of splitting the syntax of expressions into $E$- and $I$-forms has several uses.
In proof search, this syntax is useful for generating normal form proofs which, by the Curry-Howard Isomorphism~\todo{cite}, is equivalent to generating normal form programs in our setting~\todo{cite}.
Dividing the syntax in this manner also enables \emph{bidirectional type checking}~\todo{cite} where we explicitly encode when a type is an input or output in the type checking judgment.
We discuss the relevance of bidirectional type checking to program synthesis in \autoref{sec:introducing-lsyn}.

\input{figures/stlc-unit-gen-normal.tex}

\autoref{fig:stlc-unit-gen-normal} gives the rules for generating \stlcu{} terms in this restricted term language.
Because there are two syntactic forms, we introduce two judgments to synthesize $E$-($\synE$) and $I$-($\synI$) terms.
Otherwise, the rules are identical to the $e$-term generation judgment given in \autoref{fig:stlc-unit-gen}.

\section{Specifying \texorpdfstring{$λ$}{λ}-terms}
\label{sec:specifying-lambda-terms}

With well-typed term enumeration, we can efficiently search the space of \stlcu{} programs.
However, in addition to this search method, we also require some way of \emph{specifying} which particular programs we would like to find during the synthesis process.
Specification can take many forms, for example, logical properties~\todo{cite} or execution traces~\todo{cite}.
Here, we will consider specifying programs with a combination of types and examples.
The most common form of an example is the input/output pair $(v ⇒ v')$ that says that a synthesized function should produce the value $v'$ when given the value $v$.

Examples are necessarily incomplete specifications as we can only specify a finite subset of a function's behavior with a finite set of input/output pairs.
In contrast, others forms of specification, \eg, first-order logic statements, can provide a complete description of a function's behavior.
This is a significant limitation, but in practice, we can usually \emph{generalize} the behavior implied by a finite set of input/output pairs into the complete, desired function~\todo{cite}.
Furthermore, examples are typically more natural to write down than these logical statements.
In some cases these logical statements are as complex---or even more complex---then the programs we were intending to synthesize!

For example, consider the set of input/output examples specifying a function in a ML-like language:
\begin{center}
  \begin{minipage}{0.35\textwidth}
    \begin{lstlisting}
sort [] = []
sort [0] = [0]
sort [0; 1] = [1; 0]
sort [0; 1; 2] = [2; 1; 0]
    \end{lstlisting}
  \end{minipage}
\end{center}
We can infer that the user wishes to synthesize a function that sorts a list of numbers.

In contrast, consider a logical specification of how this function ought to behave:
\begin{gather*}
  \mkwd{sorted}(l) = ∀ x :: l' ⊑ l.\; ∀ y ∈ l.\; x < y
\end{gather*}
The $\mkwd{sorted}$ predicate states that for all sub-lists ($⊑$) of $l$, the head $x$ of that sub-list is least element of that sub-list.
While the logical specification precisely describes the behavior of \lstinline!sort!, it takes a bit of effort to come up with this definition of sortedness.

Finally, we use examples because they decompose naturally with types.
For example, the input/output example \lstinline!stutter [0; 1] = [1; 0]! decomposes according to the arrow type $\mkwd{int} → \mkwd{int}$.
We know that when \lstinline!stutter!'s argument is \lstinline![0; 1]!, the body of the function ought to evaluate to \lstinline![1; 0]!.
In contrast, it is not clear how to decompose $\mkwd{sorted}$ predicate according to the type of the function we are synthesizing.

To adopt example specifications into \stlcu{}, we need to account for the presence of higher-order functions.
Therefore, we adopt the following grammar of \emph{example values} $χ$ that we use as our specification:
\[
  χ \bnfdef () \bnfalt \many{v_i ⇒ χ_i}{i < n}
\]
Intuitively, for each type of \stlcu{}, we include a term that stands in as an example of that type.
In most cases, this term is simply the values for that type, \eg, $()$ for $\mkwd{Unit}$.
However, lambdas serve as a poor example value for arrow types because providing a function value is tantamount to describing exactly how the function behaves!
Instead, we make sets of input/output pairs first-class example values with \emph{partial function} terms written:
\[
  v_1 ⇒ χ_1 \bnfalt … \bnfalt v_n ⇒ χ_n
\]
A partial function introduces $n$ input/output pairs of the form $v_i ⇒ χ_i$ that specify that when the synthesized function is supplied a $v_i$, it produces a $χ_i$.
For short-hand, we write the above partial function as $\many{v_i ⇒ χ_i}{i < n}$.

As a concrete example within \stlcu{}, consider the following synonyms for the church encodings of booleans over the $\mkwd{Unit}$ type:
\begin{align*}
  \mkwd{bool}  &≡ \mkwd{Unit} → \mkwd{Unit} → \mkwd{Unit}   \\
  \mkwd{true}  &≡ λt{:}\mkwd{Unit}.\;λf{:}\mkwd{Unit}.\;t \\
  \mkwd{false} &≡ λt{:}\mkwd{Unit}.\;λf{:}\mkwd{Unit}.\;f
\end{align*}
Then the following partial function:
\begin{gather*}
  \mkwd{true} ⇒ () ⇒ () ⇒ () \bnfalt \mkwd{false} ⇒ () ⇒ () ⇒ ()
\end{gather*}
specifies the $\mkwd{if}$ function of type $\mkwd{bool} → \mkwd{Unit} → \mkwd{Unit} → \mkwd{Unit} $, usually defined as:
\[
  \mkwd{if} ≡ λb:\mkwd{bool}.\;λt{:}\mkwd{Unit}.\;λf{:}\mkwd{Unit}.\;b\;t\;f
\]

Note that this specification only \emph{partially} specifies the $\mkwd{if}$ function.
There are many other programs that this specification implies, for example, the more obvious function:
\[
  \mkwd{if'} ≡ λb:\mkwd{bool}.\;λt{:}\mkwd{Unit}.\;λf{:}\mkwd{Unit}.\;()
\]

Indeed, within \stlcu{} (which features both a base type $\mkwd{Unit}$ with a single value $()$ and strong normalization), $\mkwd{if}$ and $\mkwd{if'}$ are contextually equivalent.
That is, $\mkwd{if'}$ can be exchanged for $\mkwd{if'}$ (and vice versa) in any \stlcu{} program and the resulting program behaves identically to the original.

In \autoref{ch:simple-type-extensions} and \autoref{ch:recursion}, we enhance \stlcu{} with additional types so that we can synthesize more interesting programs.
However, for now, \stlcu{} is sufficient to establish the theoretical foundation for program synthesis that we seek.

\section{The Synthesis Problem}

Now that we have identified the sort of specification we will use during synthesis, we must describe \emph{how} we will use the specification during the synthesis process.
Our synthesis problem consists of a \emph{goal type} $τ$ and a collection of \emph{example values} $Χ = χ_1, …, χ_n$ of that type $τ$.
A solution to the synthesis problem is a program $e$ that satisfies the examples $X$.

What does it mean for a program to satisfy a set of examples?
Intuitively, if we think of 

A simple way to utilize specification is to simply enumerate terms as per \autoref{fig:stlc-unit-gen-normal} and check each to see if they satisfy the example values.
With our simple grammar of example values, this is straightforward:
\begin{itemize}
  \item At $\mkwd{Unit}$ type, the example value must be a $()$.  Therefore, the term must evaluate to $()$.
  \item At arrow type, the example value must all be partial functions $\many{v_i ⇒ χ_i}{i < m}$.
    The term $f$ that we synthesize is a function, so it suffices to check for each alternative $v ⇒ χ$ of the partial function that $(f\,v)$ evaluates to $χ$.
\end{itemize}
However, this process doesn't utilize examples at all \emph{during} term generation.
This seems like a waste because we ought to be able to refine our search based off the example values that the user provides.

\section{Introducing \texorpdfstring{$λ_{syn}$}{λsyn}}
\label{sec:introducing-lsyn}

So far, we modified the type checking judgment of \stlcu{} to create a type-directed term enumeration judgment and defined a grammar of example values to serve as our specification.
Let us combine these two into a \emph{synthesis calculus}, \lsynu{}, that synthesizes \stlcu{} terms.

\input{figures/lsyn-unit-defn.tex}

\autoref{fig:lsyn-unit-defn} gives the syntax of \lsynu{} which features an external language of standard expressions $e$ taken from \stlcu{} and an internal language that splits up expressions into introduction forms $I$ and elimination forms $E$.
The internal language is a subset of the external language; as described in \autoref{subsec:enumerating-normal-forms}, it is precisely the normal forms of the external language.

Type checking and evaluation in the external language is the same as \stlcu{}.
We provide additional rules for type checking the additional constructs introduced in \lsynu{}.
In particular, $E$ and $I$ terms are checked using bidirectional type checking~\todo{cite}.
We note that during regular type checking, $E$ terms \emph{generate} types (as an output), and $I$ terms \emph{check} against types (as an input).
To make this explicit, we separate $E$ and $I$ type checking into two separate judgments were we generate types $(E ⇒ τ)$ in the former case and check against types in $(τ ⇐ I)$ in the latter case.
For example when type checking a variable (\rulename{t-Evar}), we can extract the type of that variable from the context.
In contrast, when type checking a function (\rulename{t-Ilam}), while we know the type of the argument from the ascription, we have no information about the body, so we must check the body against a given input type.

Type checking examples values $χ$, in contrast, is straightforward.
The unit example value $()$ has type $\mkwd{Unit}$ (\rulename{t-ex-unit}).
A partial function $\many{v_i ⇒ χ_i}{i < m}$ has type $τ_1 → τ_2$ if all of $v_i$ have type $τ_1$ and all of the $χ_i$ have type $τ_2$.
When synthesizing a program, we give both a set of example values and a goal type $τ$ with the assumption that each example value has type $τ$.

The user does not provide a single example, but sets of examples $Χ$.


\subsection{Example Refinement}
\label{subsec:example-refinement}

While we have defined \emph{what} our specification is---input/output example values $χ$---we have not yet described \emph{how} to use this specification to refine our search.
A simple way to utilize specification is to simply enumerate terms and check each to see if they satisfy the example values.
With our simple grammar of example values, this is straightforward:
\begin{itemize}
  \item At $\mkwd{Unit}$ type, the example value must be a $()$.  Therefore, the term must evaluate to $()$.
  \item At arrow type, the example value must all be partial functions $\many{v_i ⇒ χ_i}{i < m}$.
    The term $f$ that we synthesize is a function, so it suffices to check for each alternative $v ⇒ χ$ of the partial function that $(f\,v)$ evaluates to $χ$.
\end{itemize}
However, this process doesn't utilize examples at all \emph{during} term generation.
This seems like a waste because we ought to be able to refine our search based off the example values that the user provides.

\begin{figure}
  \begin{center}
    \begin{gather*}
      \fbox{$Γ ⊢ τ ⇝ E$} \qquad
        \inferrule[eguess-var]
          {x:τ ∈ Γ}
          {Γ ⊢ τ ⇝ x} \qquad
        \inferrule[eguess-app]
          {Γ ⊢ E ⇝ τ_1 → τ_2 \\ Γ ⊢ τ_1 ▷ · ⇝ I}
          {Γ ⊢ τ_2 ⇝ E\;I} \\
      \fbox{$Γ ⊢ τ ▷ Χ ⇝ I$} \qquad
        \inferrule[irefine-arr]
          {X = σ_1 ↦ pf_1, …, σ_n ↦ pf_n \\\\
          Χ' = \mfun{apply}(x, σ_1, pf_1) \concat … \concat \mfun{apply}(x, σ_n, pf_n) \\\\
          x:τ_1, Γ ⊢ τ_2 ▷ Χ' ⇝ I}
          {Γ ⊢ τ_1 → τ_2 ▷ Χ ⇝ λx:τ_1.\;I} \\
        \inferrule[irefine-unit]
          {Χ = σ_1 ↦ (), …, σ_n ↦ ()}
          {Γ ⊢ 1 ▷ Χ ⇝ ()} \qquad
        \inferrule[irefine-guess]
          {Γ ⊢ τ ⇝ E \\ E ⊢ Χ}
          {Γ ⊢ τ ▷ Χ ⇝ E} \\
      \fbox{$I ⊢ Χ$} \qquad
        \inferrule[satisfies]
          {∀σ ↦ χ ∈ Χ.\;σ(I) ⟶^* v ∧ v ≃ χ}
          {I ⊢ Χ} \\
      \fbox{$v ≃ χ$} \qquad
        \inferrule[eq-unit]
          { }
          {() ≃ ()} \qquad
        \inferrule[eq-lam-pf]
          {∀i ∈ 1, …, m.\; (λx:τ.\;e)\;v_i ⟶^* v ∧ v ≃ χ_i}
          {λx:τ.\;e ≃ \many{v_i ⇒ χ_i}{i < m}} \\
      \mfun{apply}(x, σ, \many{v_i ⇒ χ_i}{i < m}) = [v_i/x]σ ↦ χ_i, …, [v_m/x]σ ↦ χ_m
    \end{gather*}
  \end{center}

  \hrule
  \caption{\lsynu{} synthesis and equivalence}
  \label{fig:lsyn-unit-synthesis}
\end{figure}


To fix this problem, we introduce \emph{example refinement} to create a full-fledged synthesis procedure within \lsynu{}.
We break up synthesis into two judgments as shown in \autoref{fig:lsyn-unit-synthesis}:
\begin{itemize}
  \item $Γ ⊢ τ ⇝ E$ (\rulename{eguess}): guess an $E$ of type $τ$.
  \item $Γ ⊢ τ ▷ Χ ⇝ I$ (\rulename{irefine}): refine and synthesize an $I$ of type $τ$ that agrees with examples $Χ$.
\end{itemize}

With $I$-terms, we are able to leverage the examples the user provides.
For example, at $\mkwd{Unit}$ type, there is only one value, $()$ of that type, so the examples values must all be $()$.
