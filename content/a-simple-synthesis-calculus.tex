Program synthesis is an undecidable problem, so when building practical synthesis tools, we must resort to approximations and heuristics to not only make the problem solvable, but also tractable.
However, this reality is frequently at odds with building a system that is understandable, as the approximations and heuristics frequently bleed together so that their individual contributions are not clear.
Ideally, we would like to build synthesis systems in such a way that we know precisely the impact of each feature and design choice.
For example, does ruling out a certain class of programs impact the completeness of the synthesis algorithm?
If so, are these programs relevant or are they safe to ignore for practical purposes?

To this end, we begin by constructing a theoretical foundation for program synthesis with types.
In typical programming language theory style, we strip away everything but the essential components of typed, functional programming languages by starting with the simply-typed lambda calculus, \stlc{}.
From \stlc{}, we build a generator for well-typed $λ$-terms and then integrate a notion of specification into the system to create a program synthesis calculus, \lsyn{}.

\lsyn{} itself is far removed from a usable implementation of a program synthesizer.
However, it allows us to deeply study how to synthesize programs with types, and its meta-theoretic properties.
For example, the synthesis judgment of \lsyn{} is sound and complete with respect to synthesis.
In successive chapters, we evolve \lsyn{} into a practical program synthesizer for ML-like programs, sacrificing some of these properties to handle advanced types and gain efficiency.

\section{Generating \texorpdfstring{$λ$}{λ}-terms}
\label{sec:generating-lambda-terms}

\begin{figure}
  \begin{center}
    \begin{tabular}{>{$}l<{$} >{$}r<{$} >{$}l<{$} l}
      τ & \bnfdef & τ_1 → τ_2 \bnfalt T & Types\\
      e & \bnfdef & x \bnfalt e_1\,e_2 \bnfalt λx{:}τ.\,e \bnfalt c & Terms \\
      v & \bnfdef & λx{:}τ.\,e \bnfalt c & Values \\
      ℰ & \bnfdef & ◻ \bnfalt ℰ\,e \bnfalt v\,ℰ & Evaluation Contexts \\
      Γ & \bnfdef & · \bnfalt x{:}τ, Γ & Typing Contexts \\
    \end{tabular} \\[12pt]
    \hrule
    \begin{gather*}
      \fbox{\mbox{$Γ ⊢ e : τ$}} \qquad
        \inferrule[t-var]
          {x{:}τ ∈ Γ}
          {Γ ⊢ x : τ} \qquad
        \inferrule[t-app]
          {Γ ⊢ e_1 : τ_1 → τ_2 \\ Γ ⊢ e_2 : τ_1}
          {Γ ⊢ e_1\,e_2 : τ_2} \\
        \inferrule[t-lam]
          {x{:}τ_1, Γ ⊢ e : τ_2}
          {Γ ⊢ λx{:}τ_1.\,e : τ_1 → τ_2} \qquad
        \inferrule[t-base]
          { }
          {Γ ⊢ c : T} \\
      \fbox{\mbox{$e ⟶ e'$}} \qquad
        \inferrule[eval-ctx]
          {e ⟶ e'}
          {ℰ[e] ⟶ ℰ[e']} \qquad
        \inferrule[eval-app]
          {}
          {(λx{:}τ.\,e)\;v ⟶ [v/x]e}
    \end{gather*}
  \end{center}

\hrule
\caption{\stlc{} definition}
\label{fig:stlc-defn}
\end{figure}


\autoref{fig:stlc-defn} gives the syntax and semantics for \stlc{} which contains the essence of a typed, functional programming language---variables, functions, and application---and their usual type checking rules.
In addition to these essentials, we also augment \stlc{} with a base type $T$ and a finite, non-empty set of constants $c_1, …, c_k$ of type $T$.
While multiple constants are not necessary---a single constant makes $T$ isomorphic to the standard $\mUnit$ type---they allow us to synthesize interesting programs.
We also define a small-step, call-by-value operational semantics for the language using evaluation contexts $ℰ$ which capture the standard of congruence rules for call-by-value evaluation.

\begin{figure}
  \begin{center}
    \begin{tikzpicture}
      \begin{semilogyaxis}[
        xlabel=Size (AST nodes),
        ylabel=Count,
        xmin=0,
        xmax=16,
        ymin=1,
        ymax=3000000,
        domain=1:15,
        samples=15,
        legend entries={All untyped, All typed,$\mkwd{Unit}$ type},
        legend pos=north west
      ]
        \addplot table[x index=0, y index=1] {data/lambda-terms-counts.data};
        \addplot table[x index=0, y index=2] {data/lambda-terms-counts.data};
        \addplot table[x index=0, y index=3] {data/lambda-terms-counts.data};
      \end{semilogyaxis}
    \end{tikzpicture}
  \end{center}

\hrule
\caption[Closed $λ$-term counts]{Number of closed untyped and typed terms and terms of type $\mkwd{Unit}$ in \stlcu{}}
\label{fig:lambda-terms-counts}
\end{figure}


To generate \stlc{} terms, we could simply enumerate terms according to the grammar given in \autoref{fig:stlc-defn}, perhaps in order of increasing term size.
We'll define the size of a term to be the number of abstract syntax tree (AST) nodes used to represent the term.
For example, the term $λx{:}T.\,y\,x$ has size 5 because the type $T$ has size one, the application $y\;x$ has size 3, and the lambda itself contributes one additional AST node.

However, simple enumeration is not very practical, especially when we have the type system at our disposal.
To see why, consider the number of closed \stlc-terms---terms generated from the empty typing context---at a given size.
\autoref{fig:lambda-terms-counts} gives the number of such untyped terms, well-typed terms, and well-typed terms at type $T$ when there are two constants $c_1$ and $c_2$ of type $T$.
For example:
\begin{itemize}
  \item $c_1\,(λx{:}T.\,x)$ is a syntactically valid, yet ill-typed term of size five.
  \item $c_2$ is a well-typed term of type $T$ and size one.
  \item $(λx{:}T → T.\,x)\,(λx{:}T.\,x)$ is a well-typed term of type $T → T$ of size nine.
\end{itemize}

Noting that the $y$-axis scale of \autoref{fig:lambda-terms-counts} is at logscale, we can see that the number of syntactically valid, but not necessarily well-typed terms is staggering.
Even in this extremely limited language, there are 11,084,176 such terms at size 15!
In contrast, if we limit our consideration to only the well-typed terms, we save an order of magnitude of work; there are only 1,439,481 well-typed terms of size 15 in \stlc.
Finally, if we consider only well-typed programs, we may as well further refine our search to programs of a particular type that we are interested in.
The savings here is even larger; if we are trying to find terms of type $T → T$, we only have to search 6,205 terms.

We could refine our term generation strategy by enumerating syntactically well-typed terms and then filtering out terms that fail to type check.
This has the benefit of ensuring that any terms that we keep around will be well-typed.
However, we still pay the time and space overhead of generating such terms and then type checking them, even if we end up throwing them away in the end.
This is undesirable in the presence of the steep combinatorial explosion of terms as their size increases, for example, \citet{grygiel-jfp-2013} show that there are approximately $2^{219}$ closed-terms in the simply-typed lambda calculus of size 50.
Rather than type check after the fact, we should \emph{integrate} type checking into term enumeration so that we only ever consider well typed terms.

To arrive at such an algorithm, we start with the type checking relation presented in \autoref{fig:stlc-defn}:
\[
Γ ⊢ e : τ.
\]
When implementing a type checker based on this relation, we note that $Γ$ and $e$ serve as inputs and $τ$ serves as either an input or output.
To derive a type-aware term enumeration system from this judgment, let's simply flip the inputs and outputs: $Γ$ and $τ$ will be inputs and the output will be an $e$.
The resulting relation, $Γ ⊢ τ ⇝ e$, describes the possible ways we can generate a term $e$ of type $τ$.

\begin{figure}
  \begin{center}
    \begin{gather*}
      \fbox{\mbox{$Γ ⊢ τ ⇝ e$}} \qquad
        \inferrule[gen-var]
          {x ∈ Γ}
          {Γ ⊢ τ ⇝ x} \qquad
        \inferrule[gen-app]
          {Γ ⊢ τ_1 → τ_2 ⇝ e_1 \\ Γ ⊢ τ_1 ⇝ e_2}
          {Γ ⊢ τ_2 ⇝ e_1\,e_2} \\
        \inferrule[gen-lam]
          {x{:}τ_1, Γ ⊢ τ_2 ⇝ e}
          {Γ ⊢ τ_1 → τ_2 ⇝ λx{:}τ_1.\,e} \qquad
        \inferrule[gen-base]
          { }
          {Γ ⊢ T ⇝ c}
    \end{gather*}
  \end{center}

\hrule
\caption{\stlc{} typed term generation}
\label{fig:stlc-gen}
\end{figure}


\autoref{fig:stlc-gen} gives the rules for \stlc{} well-typed term enumeration.
A complete derivation in this judgment corresponds to how we generate a particular term of a given type.
Therefore, searching for all the valid derivations for a particular $Γ$ and $τ$ is equivalent to enumerating all programs of type $τ$ under $Γ$.

As a concrete example, we generate the well-typed closed term $(λx{:}T → T.\,x)\,(λx{:}T.\,c_1)$ through the following derivation:
\[
\inferrule*
  {\inferrule*
    {\inferrule*
      {x{:}T → T ∈ x{:}T → T}
      {x{:}T → T ⊢ T → T ⇝ x}}
    {· ⊢ (T → T) → (T → T) ⇝ (λx{:}T → T.\,x)} \\
  \inferrule*
    {\inferrule*
      { }
      {x{:}T ⊢ T ⇝ c_1}}
    {· ⊢ (T → T) ⇝ (λx{:}T.\,c_1)}
  }
  {· ⊢ T → T ⇝ (λx{:}T → T.\,x)\,(λx{:}T.\,c_1)}
\]
Note that the derivation for term enumeration is identical to the derivation for type checking!
This is because the rules for well-typed term enumeration mirror the rules for type checking.\footnote{%
  In fact, because the type-checking and and term-enumeration judgments describe \emph{relations} which do not have inputs or outputs, they are equivalent!
  But then, why did we bother including a term-enumeration judgment at all?
  We use this term-enumeration judgment as a starting point to arrive at a synthesis judgment that will do more than relate terms and types; it will relate examples as well.
}
This observation leads to two natural properties of our term enumeration judgment:

\begin{proofenv}
  \begin{theorem}[Soundness of \stlc{} term enumeration]
  \label{thm:soundness-stlc-term-enumeration}
  If $Γ ⊢ τ ⇝ e$ then $Γ ⊢ e : τ$.
  \end{theorem}
  \begin{theorem}[Completeness of \stlc{} term enumeration]
  \label{thm:completeness-stlc-term-enumeration}
  If $Γ ⊢ e : τ$ then $Γ ⊢ τ ⇝ e$.
  \end{theorem}
  \begin{proof}
    Both theorems follow from straightforward inductions on the given term-enumeration or type-checking derivations, respectively.
  \end{proof}
\end{proofenv}

\autoref{thm:soundness-stlc-term-enumeration} states that enumerated terms are well-typed, and \autoref{thm:completeness-stlc-term-enumeration} states that we are able to enumerate the well-typed terms.
By combining both theorems, we know that well-typed term enumeration produces \emph{exactly} the set of well-typed terms, \ie, the ``All Typed'' dataset from \autoref{fig:lambda-terms-counts}.

\subsection{Enumerating Normal Forms}
\label{subsec:enumerating-normal-forms}

By restricting term generation to well-typed terms, we substantially reduce the search space of programs.
However there are still many redundant terms that we could generate.
For example, the following terms are redundant:
\begin{itemize}
  \item $(λx{:}T.\,x)\,c_1$
  \item $(λx{:}T → T.\,x)\,(λy{:}T.\,y)$.
\end{itemize}
These terms evaluate, according to our call-by-value semantics, to the values $c_1$ and $λy{:}T.\,y$, respectively.
In general, any term that is not a value, \ie, not in \emph{normal form}, is redundant with the value that it evaluates to.
We say that the two terms are \emph{$β$-equivalent}, taking the standard definition of $β$-equivalence as the smallest equivalence relation that contains our evaluation relation ($⟶$).\footnote{%
  This is true of \stlc{} because it is strongly normalizing.
  In the presence of non-termination, this does not hold because infinite loops do not have a corresponding normal form.
}

By restricting our search to these normal forms, we narrow the search space of programs even further!
Returning to \autoref{fig:lambda-terms-counts}, we see that the number of closed normal forms of size 15 and type $T → T$ is another order of magnitude smaller than the number of terms of type $T → T$ (1,489 versus 6,205 such terms).
To do this, we restrict terms so that they cannot contain pending $β$-reductions.
In \stlc{}, the only $β$-reduction available is function application which applies whenever we have a term of the form $(λx{:}τ.\,e)\;v$.
We avoid this situation by splitting expressions into two sets of sub-terms.
\begin{align*}
  E &\bnfdef x \bnfalt E\,I \\
  I &\bnfdef E \bnfalt λx{:}τ.\,I \bnfalt c
\end{align*}
$E$-forms include variables and \emph{elimination} forms, and $I$-forms include \emph{introduction} forms.
For example, lambdas introduce values of arrow type, and function application eliminates these values.
In contrast, the base type $T$ is introduced by its constants $c_1, …, c_k$ and have no corresponding elimination forms.

With this syntax, $I$ and $E$ terms are in normal form by construction!
To see this, note that function application requires that the head of a function application is an $E$, and the syntax of $E$s bottoms out at variables.
Therefore, all applications will be of the form $x\,I_1 … I_m$ where a variable is always at the head of the application.
Arguments, in contrast, are allowed to be any $I$-term which includes $E$s because $E$s are included in the definition of $I$.

This technique of splitting the syntax of expressions into introduction and elimination forms has several useful effects.
In proof search, this syntax is useful for generating normal form proofs which, by the Curry-Howard Isomorphism~\citep{william-curry-1980}, is equivalent to generating normal form programs in our setting.
Dividing the syntax in this manner also facilities a description of \emph{bidirectional type checking}~\citep{pierce-toplas-2000} where we explicitly state when a type is an input or output in the type checking judgment.
This distinction becomes very useful when we talk about refining specification in the context of full program synthesis in \autoref{sec:lsyn-a-program-synthesis-calculus}.

\begin{figure}
  \begin{center}
    \begin{gather*}
      \fbox{\mbox{$Γ ⊢ τ \synE E$}} \qquad
        \inferrule[gen-Evar]
          {x{:}τ ∈ Γ}
          {Γ ⊢ τ \synE x} \qquad
        \inferrule[gen-Eapp]
          {Γ ⊢ τ_1 → τ_2 \synE E \\ Γ ⊢ τ_1 \synI I}
          {Γ ⊢ τ_2 \synE E\;I} \\
      \fbox{\mbox{$Γ ⊢ τ \synI I$}} \qquad
        \inferrule[gen-Ielim]
          {Γ ⊢ τ \synE E}
          {Γ ⊢ τ \synI E} \qquad
        \inferrule[gen-Ilam]
          {x{:}τ_1, Γ ⊢ τ_2 \synI I}
          {Γ ⊢ τ_1 → τ_2 \synI λx{:}τ_1.\,I} \qquad
        \inferrule[gen-Ibase]
          { }
          {Γ ⊢ T \synI c}
    \end{gather*}
  \end{center}

\hrule
\caption{\stlc{} typed normal-form term generation}
\label{fig:stlc-gen-normal}
\end{figure}


\autoref{fig:stlc-gen-normal} gives the rules for generating \stlc{} terms in this restricted term language.
Because there are two syntactic forms, we introduce two judgments to synthesize $E$- and $I$-terms, $\synE$ and $\synI$, respectively.
Note that the judgments are mutually recursive because the definition of $E$ and $I$ terms is mutually recursive.
The rule \rulename{gen-Ielim} allows us to generate an $E$ where ever an $I$ is expected (because $E$s are a proper subset of the $I$s).
And the rule \rulename{gen-Eapp} generates the function as an $E$ and its argument as an $I$.
Otherwise, the rules are identical to the $e$-term generation judgment given in \autoref{fig:stlc-gen}.

\section{Specifying \texorpdfstring{$λ$}{λ}-terms}
\label{sec:specifying-lambda-terms}

With well-typed term enumeration, we can efficiently search the space of \stlc{} programs.
In addition to this search method, we also require some way of \emph{specifying} which particular programs we would like to find during the synthesis process.
Specification can take many forms, for example, logical properties~\citep{solar-lezama-thesis-2008, kuncak-pldi-2010} or execution traces~\citep{lau-thesis-2001}.
Here, we will consider specifying programs with a combination of \emph{types} and \emph{examples}, in particular, input/output pairs of the form $(v ⇒ v')$ that say a synthesized function should produce the value $v'$ when given the value $v$.

\subsection{Examples as Specification}
\label{subsec:examples-as-specification}

There are trade-offs involved in choosing examples as our means of specification over other options.
Examples typically act as incomplete specifications---for example, a finite set of input/output example pairs can only specify a finite subset of a function's behavior.
This is a significant limitation because many non-equivalent programs satisfy a set of examples, and the synthesizer must choose among them by inferring the intent of the user.
For example, if the synthesizer is provided with the single input/output example
\[
  c_1 ⇒ c_2 ⇒ c_1
\]
that states the synthesized program should return $c_1$ when given $c_1$ and $c_2$ as input, it isn't clear if the user wanted the program that always selects its first argument
\[
  λx{:}T.\,λy{:}T.\,x,
\]
or the program that always returns $c_1$,
\[
  λx{:}T.\,λy{:}T.\,c_1.
\]
or some more elaborate function.
To solve this problem, the user must specify more examples which is burdensome, or the synthesizer must resort to heuristics to choose a final program, \eg, choosing the smallest program with the fewest number of variables.
Thankfully, heuristics like this tend to work well in practice.
Furthermore, this ambiguity can be mitigated by smart interface design decisions, \eg, displaying the top five synthesis results rather than a singular result and allowing the user to choose.

In exchange for being incomplete, examples can be less burdensome on the user to specify.
Because examples are typically concrete instances of a function's behavior, they allow the user to reason more about \emph{what} their function ought to do rather than \emph{how} the function ought to go about doing it.
This is favorable because the ``how'' is usually the sticking point for a user to create a program on their own, whether it is because they do not understand the syntax of programming language they must use or the algorithms necessary to produce the desired behavior.

For example, consider the set of input/output examples specifying a function in a language extended with natural numbers and lists:
\begin{align*}
  []  &⇒ [] \\
  [0] &⇒ [0] \\
  [0, 1] &⇒ [1, 0] \\
  [0, 1, 2] &⇒ [2, 1, 0]
\end{align*}
From this limited set of examples, the user's intent seems obvious: they want to synthesize a function that sorts a list of natural numbers.
In contrast, consider a logical specification of how this function ought to behave:
\begin{gather*}
  \mkwd{sorted}(l) = ∀ x :: l' ⊑ l.\; ∀ y ∈ l.\; x < y
\end{gather*}
The $\mkwd{sorted}$ predicate states that for all sub-lists ($⊑$) of $l$, the head $x$ of that sub-list is the least element of that sub-list.
While this logical specification precisely describes the behavior of any function that sorts a list, it takes a bit of ingenuity to come up with this specification.

Finally and most importantly, we use examples because they decompose naturally with types.
For example, the input/output example $[0, 1] ⇒ [1, 0]$ decomposes according to the arrow type $\mlist → \mlist$.
We know that when the synthesized function's argument has the value $[0, 1]$, the body of that function must evaluate to $[1, 0]$.
In contrast, it is not clear how to decompose the $\mkwd{sorted}$ predicate according to the type of the function we are synthesizing.

\subsection{Example Specification in \texorpdfstring{\stlc}{λ→}}
\label{subsec:example-specification-in-stlc}

To adopt example specifications into \stlc{}, we need to account for the presence of higher-order functions.
Therefore, we adopt the following grammar of \emph{example values} $χ$ that we use as our specification:
\[
  χ \bnfdef c \bnfalt \many{v_i ⇒ χ_i}{i < m}
\]
For each type of \stlc{}, we include a term that stands in as an example of that type.
As we extend our language with additional features, we'll find that most of the time this term is simply the values for that type, \eg, the constants $c_k$ for base type $T$.
However, lambdas serve as a poor example value for arrow types because providing a function value is tantamount to describing exactly how the function ought to behave!
Instead, we make sets of input/output pairs first-class example values with \emph{partial function} terms written:
\[
  v_1 ⇒ χ_1 \bnfalt … \bnfalt v_n ⇒ χ_m
\]
A partial function introduces $n$ input/output pairs of the form $v_i ⇒ χ_i$ that specify that when the synthesized function is supplied a $v_i$, it produces a $χ_i$.
Like $→$, $⇒$ associates to the right, and as a result, functions of $k$ arguments are represented in curried style, $v_1 ⇒ … ⇒ v_k ⇒ χ$.
Because of this, we restrict the right-hand side of a $⇒$ to be an example value.
This ensures that lambdas never appear as the \emph{goal} of a synthesis problem.
In contrast, the left-hand side of a $⇒$ is a regular value which includes lambdas (and not partial functions).
This means that when we give examples for higher-order functions, we provide function values for the arguments of those functions.
We disallow partial functions as examples for higher-order arguments as doing so has surprising effects on the metatheory of the system; we explore this in more detail in~\autoref{ch:metatheory-of-mlsyn} where we lift this restriction to handle more advanced language features.
For shorthand, we write the above partial function as $\many{v_i ⇒ χ_i}{i < m}$ where the partial function contains $m$ input/output pairs.

As a concrete example within \stlc{}, consider the following synonyms for the church encodings of booleans over base type $T$:
\begin{align*}
  \mBool  &≝ T → T → T   \\
  \mtrue  &≝ λt{:}T.\,λf{:}T.\,t \\
  \mfalse &≝ λt{:}T.\,λf{:}T.\,f
\end{align*}
Then the following partial function:
\begin{gather*}
  \mtrue ⇒ c_1 ⇒ c_2 ⇒ c_1 \bnfalt \mfalse ⇒ c_1 ⇒ c_2 ⇒ c_2
\end{gather*}
specifies the $\mkwd{if}$ function of type $\mBool → T → T → T$, usually defined as:
\[
  \mkwd{if} ≝ λb{:}\mBool.\,λt{:}T.\,λf{:}T.\,b\,t\,f
\]
Note that the convenience of specifying arguments to higher-order functions as lambdas allows us to use $\mtrue$ and $\mfalse$ in the input/output examples as expected.

\section{\texorpdfstring{$λ_{syn}$}{λsyn}: A Program Synthesis Calculus}
\label{sec:lsyn-a-program-synthesis-calculus}

So far, we modified the type checking judgment of \stlc{} to create a type-directed term enumeration judgment and defined a grammar of example values to serve as our specification.
Let's combine these two into a \emph{synthesis calculus}, \lsyn{}, that synthesizes \stlc{} terms.

\subsection{Integrating Search and Specification}
\label{subsec:integrating-search-and-specification}

How do we use our example values in tandem with term enumeration?
A simple strategy is to simply enumerate terms as per \autoref{fig:stlc-gen-normal} and check each term to see if it satisfies the example values.
However, we waste effort with this approach because example values ought to help rule out programs \emph{during} term generation rather than help us filter programs after the fact.
For example, if we have following input/output pairs in a partial function:
\[
… \bnfalt c_1 ⇒ c_2 \bnfalt c_2 ⇒ c_1 \bnfalt …
\]
We should never need to generate the constant functions $λx{:}T.\,c_1$ and $λx{:}T.\,c_2$ because the examples rule them out.

To develop a more robust strategy that uses examples in a productive way, let's imagine how we might synthesize a function by hand using input/output examples.
Consider synthesizing the $\mkwd{if}$ function of type $\mBool → T → T → T$ that we discussed in \autoref{subsec:example-specification-in-stlc} where $\mBool$ is shorthand for the type $T → T → T$.
We were given the partial function
\[
  \mtrue → c_1 → c_2 → c_1 \bnfalt \mfalse → c_1 → c_2 → c_2
\]
as our example value with $\mtrue$ and $\mfalse$ of type $\mBool$ serving as shorthand for $λt{:}T.\,λf{:}T.\,t$ and $λt{:}T.\,λf{:}T.\,f$, respectively.

Assuming that we have no top-level bindings to work with (\ie, we are operating in a closed context), we know from the desired goal type, $\mBool → T → T → T$, that the synthesized program should have the shape:
\[
  λb{:}\mBool.\,λt{:}T.\,λf{:}T.\,◼
\]
where the body of the function, denoted by the hole $◼$, must be filled in with an expression of type $T$.
When considering what to fill in for $◼$, we note that our two input/output examples from our partial function give rise to two possible states:
\begin{align*}
  b = \mtrue, t = c_1, f = c_2, ◼ = c_1  \\
  b = \mfalse, t = c_1, f = c_2, ◼ = c_2
\end{align*}
Each state or \emph{example world} corresponds to an imaginary execution of the function being synthesized using each of the input/output examples.
The first world corresponds to the example where we select the $\mkwd{True}$ argument, and the second world corresponds to the example where we select the $\mkwd{False}$ argument.
In each example world, we keep track of the value each argument is bound to as well as what the function should produce: $◼ = c_1$ and $◼ = c_2$, respectively.

With all this information, the correct choice of program to fill in the hole is clear.
The term $b\,t\,f$ has the correct type and satisfies our examples because we can substitute the values bound to the variables in each example world and evaluate to obtain
\begin{align*}
  \mtrue\,c_1\,c_2 ⟶^* c_1 \\
  \mfalse\,c_1\,c_2 ⟶^* c_2
\end{align*}
which are the example values we required that the term should evaluate to.

\subsection{Introducing \texorpdfstring{$λ_{syn}$}{λsyn}}
\label{subsec:introducing-lsyn}

\begin{figure}
  \begin{center}
    \begin{tabular}{>{$}l<{$} >{$}r<{$} >{$}l<{$} l}
      τ  & \bnfdef & τ_1 → τ_2 \bnfalt 1 & Types\\
      e  & \bnfdef & x \bnfalt e_1\;e_2 \bnfalt λx:τ.\;e \bnfalt () & Terms \\
      v  & \bnfdef & λx:τ.\;e \bnfalt () & Values \\
      ℰ  & \bnfdef & ◻ \bnfalt ℰ\;e \bnfalt v\;ℰ & Evaluation Contexts \\
      Γ  & \bnfdef & · \bnfalt x:τ, Γ & Typing Contexts \\
      \\
      E  & \bnfdef & x \bnfalt E\;I & Elimination Terms \\
      I  & \bnfdef & E \bnfalt λx:τ.\;I \bnfalt () & Introduction Terms \\
      σ  & \bnfdef & · \bnfalt [v/x]σ & Evaluation Environments \\
      χ  & \bnfdef & () \bnfalt pf & Example Values \\
      pf & \bnfdef & \many{v_i ⇒ χ_i}{i < m} & Partial Functions \\
      Χ  & \bnfdef & · \bnfalt σ ↦ χ, Χ &  Example Worlds
    \end{tabular} \\[12pt]
    \hrule
    \begin{gather*}
      \fbox{$Γ ⊢ e : τ$}\;\fbox{$e⟶ e'$} \qquad
        \text{(Same as \stlc{})} \\
      \fbox{$Γ ⊢ E ⇒ τ$} \qquad
        \inferrule[t-Evar]
          {x:τ ∈ Γ}
          {Γ ⊢ x ⇒ τ} \qquad
        \inferrule[t-Eapp]
          {Γ ⊢ E ⇒ τ_1 → τ_2 \\ Γ ⊢ I ⇐ τ_1}
          {Γ ⊢ E\;I ⇒ τ_2} \\
      \fbox{$Γ ⊢ I ⇐ τ$} \qquad
        \inferrule[t-Ilam]
          {x:τ_1, Γ ⊢ I ⇐ τ_2}
          {Γ ⊢ λx:τ_1.\;I ⇐ τ_1 → τ_2} \qquad
        \inferrule[t-Iunit]
          { }
          {Γ ⊢ () ⇐ 1} \\
      \fbox{$Γ ⊢ χ : τ$} \qquad
        \inferrule[t-ex-unit]
          { }
          {Γ ⊢ () : \mkwd{Unit}} \qquad
        \inferrule[t-ex-pf]
          {\many{Γ ⊢ v_i : τ_1}{i < m} \\ \many{Γ ⊢ χ_i : τ_2}{i < m}}
          {Γ ⊢ \many{v_i ⇒ χ_i}{i < m} : τ_1 ⇒ τ_2} \\
      \fbox{$Γ ⊢ σ$} \qquad
        \inferrule[t-env-empty]
          { }
          {Γ ⊢ ·} \qquad
        \inferrule[t-env-cons]
          {x:τ ∈ Γ \\ ⊢ v : τ \\ Γ ⊢ σ}
          {Γ ⊢ [v/x]σ} \\
      \fbox{$Γ ⊢ Χ : τ$} \qquad
        \inferrule[t-ex-empty]
          { }
          {Γ ⊢ · : τ} \qquad
        \inferrule[t-ex-cons]
          {Γ ⊢ σ \\ Γ ⊢ χ : τ \\ Γ ⊢ Χ : τ}
          {Γ ⊢ σ ↦ χ, Χ : τ}
    \end{gather*}
  \end{center}

  \hrule
  \caption{\lsyn{} syntax and typechecking}
  \label{fig:lsyn-defn}
\end{figure}


In the previous example, we performed a mix of \emph{symbolic evaluation} and \emph{type-directed refinement} to synthesize the program.
We now formalize this process within \lsyn{}, a core calculus for program synthesis with types.

\autoref{fig:lsyn-defn} gives the syntax of \lsyn{} which features an external language of standard expressions $e$ taken from \stlc{} and an internal language that splits up expressions into introduction forms $I$ and elimination forms $E$.
The internal language is a subset of the external language.
As described in \autoref{subsec:enumerating-normal-forms}, it is precisely the normal forms of the external language that we synthesize.

Type checking and evaluation in the external language is the same as \stlc{}.
We provide additional rules for type checking the additional constructs introduced in \lsyn{}.
In particular, $E$ and $I$ terms are checked in a bidirectional type checking style~\citep{pierce-toplas-2000}.
We note that during regular type checking, $E$ terms \emph{generate} types (as an output), and $I$ terms \emph{check} against types (as an input).
To make this explicit, we separate $E$ and $I$ type checking into two separate judgments where we generate types $(E ⇒ τ)$ in the former case and check against types in $(I ⇐ τ)$ in the latter case.
For example when type checking a variable (\rulename{t-Evar}), we can extract the type of that variable from the context.
In contrast, when type checking a function (\rulename{t-Ilam}), while we know the type of the argument from the ascription, we have no information about the body, so we must check the body against a given input type.
Type checking examples values $χ$, in contrast, is straightforward.
Constants $c_k$ all have type $T$ (\rulename{t-ex-base}), and a partial function $\many{v_i ⇒ χ_i}{i < m}$ has type $τ_1 → τ_2$ if all of $v_i$ have type $τ_1$ and all of the $χ_i$ have type $τ_2$ (\rulename{t-ex-pf}).

\subsection{Example Worlds}
\label{subsec:example-worlds}

Example worlds were a critical component of how we synthesized the $\mkwd{if}$ function in \autoref{subsec:integrating-search-and-specification}.
Recall that each example world contained not only an example value that the synthesized program needed to evaluate to, but also a collection of values that each free variable of the program takes on in that example world.
To codify this, we define an example world as a pair $σ ↦ χ$ of a \emph{goal example value} $χ$ and an \emph{environment} $σ$ that maps variables to values.
A collection of these example worlds is denoted with $Χ$, and when it is unambiguous to do so, we refer to these collections of example worlds as our ``examples''.

When lifting typechecking of example values to example worlds, written $Γ ⊢ Χ : τ$, we ensure (via \rulename{t-exw-cons}) that for each pair $σ ↦ χ$ that $χ$ has type $τ$ but also that $σ$ is well-typed.
For an environment to be well-typed, it suffices that for each binding in $σ$ to be well-typed according to the type recorded in $Γ$ (\rulename{t-env-cons}).
In addition, \rulename{t-exw-cons} requires that each $σ$ contains a binding for each variable bound in $Γ$.
This, coupled with \rulename{t-env-cons}, gives us the following consistency principle for example worlds:
\begin{proofenv}
  \begin{lemma}[Consistency of Example Worlds]
    \label{lem:consistency-of-example-worlds}
    If $Γ ⊢ Χ : τ_1$, then $∀σ ↦ χ ∈ Χ.\,x{:}τ \in Γ ↔ ([x/v] \in σ ∧ Γ ⊢ v : τ)$.
  \end{lemma}
  \begin{proof}
    \rulename{t-exw-cons} enforces the $→$ direction, and \rulename{t-env-cons} enforces the $←$ direction.
  \end{proof}
\end{proofenv}
Because our synthesis strategy requires evaluation, \autoref{lem:consistency-of-example-worlds} is necessary to ensure that every free variable has a (well-typed) value.

\subsection{Synthesis in \texorpdfstring{\lsyn{}}{λsyn}}
\label{subsec:synthesis}

\begin{figure}
  \begin{center}
    \begin{gather*}
      \fbox{$Γ ⊢ τ ⇝ E$} \qquad
        \inferrule[eguess-var]
          {x{:}τ ∈ Γ}
          {Γ ⊢ τ ⇝ x} \qquad
        \inferrule[eguess-app]
          {Γ ⊢ E ⇝ τ_1 → τ_2 \\ Γ ⊢ τ_1 ▷ · ⇝ I}
          {Γ ⊢ τ_2 ⇝ E\;I} \\
      \fbox{$Γ ⊢ τ ▷ Χ ⇝ I$} \qquad
        \inferrule[irefine-arr]
          {X = σ_1 ↦ pf_1, …, σ_n ↦ pf_n \\\\
            Χ' = \mfun{apply}(x, σ_1, pf_1) \concat … \concat \mfun{apply}(x, σ_n, pf_n) \\\\
            x{:}τ_1, Γ ⊢ τ_2 ▷ Χ' ⇝ I}
          {Γ ⊢ τ_1 → τ_2 ▷ Χ ⇝ λx{:}τ_1.\,I} \\
        \inferrule[irefine-base]
          {Χ = σ_1 ↦ c_k, …, σ_n ↦ c_k}
          {Γ ⊢ T ▷ Χ ⇝ c_k} \qquad
        \inferrule[irefine-guess]
          {Γ ⊢ τ ⇝ E \\ E ⊢ Χ}
          {Γ ⊢ τ ▷ Χ ⇝ E} \\
      \fbox{$I ⊢ Χ$} \qquad
        \inferrule[satisfies]
          {∀σ ↦ χ ∈ Χ.\;σ(I) ⟶^* v ∧ v ≃ χ}
          {I ⊢ Χ} \\
      \fbox{$v ≃ χ$} \qquad
        \inferrule[eq-ctor]
          { }
          {c_k ≃ c_k} \qquad
        \inferrule[eq-lam-pf]
          {∀i ∈ 1, …, m.\; (λx{:}τ.\,e)\;v_i ⟶^* v ∧ v ≃ χ_i}
          {λx{:}τ.\,e ≃ \many{v_i ⇒ χ_i}{i < m}} \\
      \mfun{apply}(x, σ, \many{v_i ⇒ χ_i}{i < m}) = [v_1/x]σ ↦ χ_1, …, [v_m/x]σ ↦ χ_m
    \end{gather*}
  \end{center}
  \hrule
  \caption{\lsyn{} synthesis and equivalence}
  \label{fig:lsyn-synthesis}
\end{figure}


In \lsyn{}, we further refine the term enumeration judgment we derived in \autoref{sec:generating-lambda-terms} to create a full-fledged synthesis judgment that takes advantage of example values as specification.
We break up synthesis into two judgments as shown in \autoref{fig:lsyn-synthesis}:
\begin{itemize}
  \item $Γ ⊢ τ ⇝ E$ (\rulename{eguess}): guess an $E$ of type $τ$.
  \item $Γ ⊢ τ ▷ Χ ⇝ I$ (\rulename{irefine}): refine and synthesize an $I$ of type $τ$ that agrees with examples $Χ$.
\end{itemize}
The synthesis problem in \lsyn{} can be characterized as follows:
\begin{quote}
Given a context $Γ$, goal type $τ$, and examples $Χ$ where $Γ ⊢ X : τ$, synthesize a program $I$ of type $τ$ that satisfies the examples $Χ$.
\end{quote}
The two mutually recursive judgments combine to form a non-deterministic synthesis system where a complete \rulename{irefine} derivation corresponds to a solution of this synthesis problem.

\paragraph{Refinement}
In \autoref{subsec:introducing-lsyn}, we observed that distinguishing between introductory forms $I$ and elimination forms $E$ helped us to identify when a type should be treated as an input or output during typechecking.
This insight also applies to how we treat examples in tandem with term generation.
With introduction forms, the shape of the term and its corresponding type dictate how we refine example values.
For example, at base type, our example values consist of constants.
\rulename{irefine-base} says that we can synthesize a particular constant $c_k$ if the examples all agree on that constant, \ie, the goal example value of each example world is $c_k$.
Note that if the example worlds contain differing constants, then this rule does not apply, and we would need to synthesize some other expression instead.

In contrast, at arrow type our example values consist of a set of partial functions.
\rulename{irefine-arr} says that we can synthesize a lambda if we are able to synthesize its body.
To understand how \rulename{irefine-arr} refines examples, first consider the case where $Χ$ is a single example world, $σ ↦ \many{v_i ⇒ χ_i}{i < m}$.
The example value is a single partial function consisting of $m$ input/output examples.
In this situation, the $\mfun{apply}$ meta-function generates a collection of $m$ new example worlds.
The goal example value of each world is the right-hand side of each input/output example, $χ_i$.
The environment of each new world consists of the original environment $σ$ extended with a binding for the lambda's variable, $x$.
The value bound to that variable is the left-hand side of each input/output example, $v_i$.
We then use these new example worlds to synthesize the body of the lambda.

In general, $Χ$ may contain $n$ example worlds $σ_1 ↦ \mpf_1, …, σ_n ↦ \mpf_n$.
When there are multiple example worlds, we simply $\mfun{apply}$ each of the example worlds to generate a collection $Χ_1, …, Χ_n$ of example worlds and then concatenates them all together to create the final collection of example worlds $Χ'$ to be used to synthesize the body of the lambda.
This occurs if the user specifies multiple partial functions rather than a single partial function.

Surprisingly, there is a subtle semantic distinction between a collection of partial functions and a single partial function.
To see this, consider synthesizing at goal type $T → T$ with the examples:
\begin{align*}
  \mpf_1 &= σ_1 ↦ c_1 ⇒ c_2 \bnfalt c_3 ⇒ c_4 \\
  \mpf_2 &= σ_2 ↦ c_5 ⇒ c_6 \bnfalt c_7 ⇒ c_8 \bnfalt c_9 ⇒ c_{10}.
\end{align*}
If we apply \rulename{irefine-arr}, the resulting set of example worlds that we use to synthesize the body of lambda is:
\begin{align*}
  Χ' =\;& [x/c_1]σ_1 ↦ c_2, [x/c_3]σ_1 ↦ c_4, \\
      \;& [x/c_5]σ_2 ↦ c_6, [x/c_7]σ_2 ↦ c_8, [x/c_9]σ_2 ↦ c_{10}.
\end{align*}
In contrast, if we combined the two examples worlds into a single partial function,
\begin{align*}
  ρ = σ ↦ c_1 ⇒ c_2 \bnfalt c_3 ⇒ c_4 \bnfalt c_5 ⇒ c_6 \bnfalt c_7 ⇒ c_8 \bnfalt c_9 ⇒ c_{10},
\end{align*}
then applying \rulename{irefine-arr} results in the following examples,
\begin{align*}
  Χ' =\;& [x/c_1]σ ↦ c_2, [x/c_3]σ ↦ c_4, \\
      \;& [x/c_5]σ ↦ c_6, [x/c_7]σ ↦ c_8, [x/c_9]σ ↦ c_{10}.
\end{align*}
The only difference between the two cases is the original environments, $σ_1$ and $σ_2$ versus $σ$, that we extend when deriving $Χ'$.
The initial environments shared among all the example worlds ought to be identical (as the values for the free variables of the program must be the same), so in our simply-typed setting, there is no practical difference between a collection of partial functions and a single partial function.
However, in~\autoref{ch:recursion}, we explore changes to \rulename{irefine-arr} that make this no fact longer true.

\paragraph{Guessing}
In contrast, with $E$-terms, we are unable to refine the examples in this manner.
This is because an $E$-term can be generated at any type and the shape of the $E$-term alone does not determine how we can decompose examples in a meaningful way.
In particular, say that we try to synthesize a function application and choose a particular term $E$ with some function type $τ_1 → τ_2$.
When synthesizing $I$, we would like to pass refined examples $Χ'$ such that $E\,I$ satisfies $Χ$.
However, this requires reasoning about the behavior of $E$, an arbitrary function, which is difficult to do.
In the presence of richer language features, \eg, recursion, this is undecidable.

Therefore, for $E$-terms, the \rulename{eguess} judgment resorts to term enumeration as developed in \autoref{subsec:enumerating-normal-forms}.
The rules for generating variables (\rulename{eguess-var}) and application (\rulename{eguess-app}) merely generate a term of an appropriate type.
Note that when generating the argument to an application $I$ in \rulename{eguess-app}, we pass the \emph{empty} set of examples to the \rulename{irefine} judgment.
When $Χ$ is empty, you can see that the \rulename{irefine} judgment degenerates to simple term enumeration.

However, we are not content generating any $E$-term of an appropriate type.
We want such a term that satisfies the provided examples.
The bridge rule \rulename{irefine-guess} enforces this by not only generating an $E$-term, but also explicitly ensuring that $E$ satisfies $Χ$, written $I ⊨ Χ$ (noting that $E$s are a proper subset of the $I$s).
This satisfaction judgment ensures that, for each example world, evaluating $I$ under that environment results in a value $v$ that is compatible to the goal example value for that world.
Environment or substitution application, written $σ(I)$, produces an external language expression $e$ suitable for evaluation.

This compatibility relation%
\footnote{%
  Because the compatibility relation is between two different syntactic classes, it is not an equivalence as it cannot be reflexive.
  Although, we are trying to equate a value and an example value as ``equal'' with the relation.
},
written $v ≃ χ$, only needs to compare between a value $v$ and some goal example value $χ$.
At base type, $v$ and $χ$ are constants so it suffices to ensure they are the same constant (\rulename{eq-ctor}).
At arrow type, $v$ is a lambda and $χ$ is a partial function.
To ensure compatibility, it is sufficient to check for each input/output pair of the partial function that running $v$ on the input of the pair produces a value compatible with the output of the pair (\rulename{eq-lam-pf}).

\subsection{Synthesis Examples}
\label{subsec:synthesis-examples}

To better understand how \lsyn{} operates, let's work through a number of examples.

\begin{example}
  First, consider the degenerate case where the example set is empty ($Χ = ·$).
  We claimed in \autoref{subsec:synthesis} that the \rulename{irefine} judgment degenerates to raw term enumeration when the example set is empty.
  To see that this is true, let's look at how each \rulename{irefine} rule behaves with no examples.
  At arrow type, \rulename{irefine-arr} applies and passes along the empty example set ($Χ' = ·$) when synthesizing the body of the lambda.
  At base type, \rulename{irefine-base} applies and allows us to synthesize any constant $c_k$ since $Χ$ is empty.
  Finally, \rulename{irefine-guess} allows us to synthesize any $E$ because $E ⊨ ·$ holds vacuously.
\end{example}

\begin{example}
  Consider the example $X = · ↦ c_1 ⇒ c_1 \bnfalt c_2 ⇒ c_2$ with goal type $T → T$ in the empty context.
  The following is a valid synthesis derivation of the identity function:
  \[
    \inferrule*[Right=irefine-arr]
    {\inferrule*[Right=irefine-guess]
      {\inferrule*[lab=eguess-var]
        {x{:}T ∈ x{:}T}
        {x{:}T ⊢ T ⇝ x} \\
       \inferrule*[lab=satisfies]
        {[c_1/x](x) ⟶^* c_1 ∧ [c_2/x](x) ⟶^* c_2}
        {x ⊨ [c_1/x] ↦ c_1, [c_2/x] ↦ c_2}}
      {x{:}T ⊢ T ▷ [c_1/x] ↦ c_1, [c_2/x] ↦ c_2 ⇝ x}}
    {· ⊢ T → T ▷ · ↦ c_1 ⇒ c_1 \bnfalt c_2 ⇒ c_2 ⇝ λx{:}T.\,x}
  \]
  With the way the synthesis judgments are set up, we always make a number of $\rulename{irefine}$ derivations ending in a number of $\rulename{eguess}$ derivations.
  This observation leads to a critical implementation optimization, \emph{refinement trees}, that we explore in \autoref{ch:implementation}.
\end{example}

\begin{example}
  Consider under-constraining the example set from the previous example: $Χ = c_1 ⇒ c_1$ with goal type $T → T$ in the empty context.
  We can now synthesize the constant function with a simpler derivation:
  \[
    \inferrule*[Right=irefine-arr]
    {\inferrule*[Right=irefine-base]
      {Χ = [c_1/x] ↦ c_1}
      {x{:}T ⊢ T ▷ [c_1/x] ↦ c_1 ⇝ c_1}}
    {· ⊢ T → T ▷ · ↦ c_1 ⇒ c_1 ⇝ λx{:}T.\,c_1}
  \]
  Note that the identity function is still derivable in this context:
  \[
    \inferrule*[Right=irefine-arr]
    {\inferrule*[Right=irefine-guess]
      {\inferrule*[lab=eguess-var]
        {x{:}T ∈ x{:}T}
        {x{:}T ⊢ T ⇝ x} \\
       \inferrule*[lab=satisfies]
        {[c_1/x](x) ⟶^* c_1}
        {x ⊨ [c_1/x] ↦ c_1}}
      {x{:}T ⊢ T ▷ [c_1/x] ↦ c_1 ⇝ x}}
    {· ⊢ T → T ▷ · ↦ c_1 ⇒ c_1 ⇝ λx{:}T.\,x}
  \]
  How might we choose one program over the other?
  While both programs are the same size, we may choose the program with the smaller derivation tree (the constant function), or we might choose the program that uses the most variables (the identity function).
  With our synthesis relation, we are only concerned with ensuring that every satisfying program has a corresponding derivation in the system, so this ambiguity is irrelevant.
  However, in a synthesis procedure, we will need to disambiguate between these programs in some manner, a topic we revisit in \autoref{ch:implementation}.
\end{example}

\begin{example}
  Next, consider synthesizing the $\mkwd{if}$ function from \autoref{subsec:example-specification-in-stlc}.
  Again we use the shorthand:
  \begin{align*}
    \mBool  &≝ T → T → T   \\
    \mtrue  &≝ λt{:}T.\,λf{:}T.\,t \\
    \mfalse &≝ λt{:}T.\,λf{:}T.\,f.
  \end{align*}
  Here, we are trying to synthesize a program of type $\mBool → T → T → T$ with the example set
  \begin{align*}
    Χ = · ↦ \mtrue ⇒ c_1 ⇒ c_2 ⇒ c_1, · ↦ \mfalse ⇒ c_1 ⇒ c_2 ⇒ c_2.
  \end{align*}

  We can synthesize the desired $\mkwd{if}$ function in \lsyn{} as follows.
  First we can apply \rulename{irefine-arr} three times to arrive at the synthesis state:
  \[
    b{:}\mBool, t{:}T, f{:}T ⊢ T ▷ Χ ⇝ ◼
  \]
  where the example set $Χ$ is:
  \begin{align*}
    [\mtrue/b][c_1/t][c_2/f] &↦ c_1 \\
    [\mfalse/b][c_1/t][c_2/f] &↦ c_2
  \end{align*}
  Now, we apply \rulename{irefine-guess} to generate the application.
  The following \rulename{eguess} derivation generates the application itself (with $Γ = b{:}\mBool, t{:}T, f{:}T$).
  \[
    \inferrule*[Right=eguess-app]
      {\inferrule*[Left=eguess-app]
        {\inferrule*[Left=eguess-var]
          {b{:}T → T → T ∈ Γ}
          {Γ ⊢ T → T → T ⇝ b} \\
         \inferrule*[lab=eguess-var]
          {t{:}T ∈ Γ}
          {Γ ⊢ T ⇝ t}
        }
        {Γ ⊢ T → T ⇝ b\;t} \\
      \inferrule*[Right=eguess-var]
        {f{:}T ∈ Γ}
        {Γ ⊢ T ⇝ f}
      }
      {Γ ⊢ T ⇝ b\;t\;f}
  \]
  Finally, we verify that $b\;t\;f$ satisfies the examples:
  \begin{gather*}
    [\mtrue/b][c_1/t][c_2/f](b\;t\;f) ⟶^* c_1 \\
    [\mfalse/b][c_1/t][c_2/f](b\;t\;f) ⟶^* c_2.
  \end{gather*}
\end{example}

\begin{example}
  For our last example, consider what happens if we provide contradictory examples.
  Suppose that we are synthesizing a program of type $T → T$ and provide the partial function $c_1 ⇒ c_1 \bnfalt c_1 ⇒ c_2$.
  There is no function that we can synthesize that satisfies this specification as $c_1$ is mapped to two distinct outputs.
  To see this within \lsyn{}, note that \rulename{irefine-ctor} will never apply because the goal example values are different, and \rulename{irefine-guess} will never succeed because we will never be able to generate an $E$ that satisfies both $c_1$ and $c_2$ given that the argument to the function is bound to $c_1$.
  Consequently there are no valid synthesis derivations in \lsyn{} for contradictory examples.

  Detecting such contradictory examples turns out to be tricky and undecidable given a sufficiently rich language.
  At first glance, it seems like a simple syntactic check is sufficient: check to see that the partial functions involved map similar inputs to similar outputs.
  However, because we allow lambdas as inputs to partial functions, this problem reduces to deciding function equivalence.
  Furthermore, even this check is insufficient.
  For example, consider the worlds $[c_1/x] ↦ c_1 ⇒ c_1, [c_2/x] c_1 ⇒ c_2$.
  While the partial functions map $c_1$ to distinct outputs, they still do not contradict each other because we fulfill the eventual goal example value with the variable $x$.
  Thus, we must decide not only equivalence between example values but also environments.
  Throughout this work, we assume that examples given by the user are non-contradictory, leaving detection to future work.
\end{example}

\section{Related Work}
\label{sec:related-work-lsyn}

In this chapter, we have demonstrated the fundamentals of program synthesis with types.
With these fundamentals established, we now make technical comparisons of this program synthesis style with the prior approaches that we discussed in \autoref{sec:the-landscape-of-program-synthesis}.
In particular, previous work has explored program synthesis techniques using both types and examples as modes of specification, but not in the type-theoretic style that we propose.
With our foundations, we can explain the salient features of these systems in terms of type theory and our own work.

\subsection{Example Refinement}

In \lsyn{}, we make a distinction between the terms that allow example refinement---introduction terms---and the terms that do not---elimination forms.
Ideally, we would allow example refinement through all the terms of our language.
However, this is is difficult to do for our elimination forms.
In particular, recall \lsyn{}'s rule for synthesizing function application:
\[
  \inferrule[eguess-app]
    {Γ ⊢ τ_1 → τ_2 ⇝ E \\ Γ ⊢ τ_1 ▷ · ⇝ I}
    {Γ ⊢ τ_2 ⇝ E\,I} \\
\]
To refine examples through the function application $E\,I$, we may guess a particular function $E$, refine the examples in some manner, and feed them to the synthesis sub-problem for $I$.
The examples should be refined in such a way that we synthesize an $I$ such that $E\,I$ satisfies the original set of examples.
The problem is that, in general, this requires that we invert the behavior of $E$ to discover how to refine the examples to satisfy this property.
For arbitrary $E$, this is impossible to do as the function may not have an inverse (\ie, it is surjective but not a bijection).

However, we are able to make progress if we are able to make assumptions about the possible set of functions that can appear in an application.
These sorts of assumptions are made by several \emph{combinator-based synthesis systems} such as $λ^2$~\citep{feser-pldi-2015} and \textsc{FlashExtract}~\citep{le-pldi-2014} to push examples through the application of a fixed set of functions.
For example, suppose that we are synthesizing a function with the single input/output example:
\[
  [0, 1, 2, 3] ⇒ [1, 2, 3, 4].
\]
Then, because the behavior of the standard $\mkwd{map}$ function is
\[
  \begin{array}{l}
    \mlet\;\mrec\;\mkwd{map}\;f\;l = \\
    \⇥{1}\mmatch\;l\;\mwith \\
    \⇥{1}\bnfalt [] → [] \\
    \⇥{1}\bnfalt x :: xs → f\;x :: \mkwd{map}\;f\;xs
  \end{array}
\]
we know that we can synthesize a program of the form $λl{:}\mlist.\,\mkwd{map}\,◼\,l$ where the hole is a function of type $\mnat → \mnat$ with refined examples:
\[
  0 ⇒ 1 \bnfalt 1 ⇒ 2 \bnfalt 2 ⇒ 3 \bnfalt 3 ⇒ 4.
\]
Both $λ^2$ and \textsc{FlashExtract} admit variants of this rule for $\mkwd{map}$ and other functional combinators.

Such rules are admissible in our system.
For example, here is an equivalent rule for $I$-refining $\mkwd{map}$ in our system extended with recursion (\autoref{ch:recursion}) and polymorphism (\autoref{ch:polymorphism}).
\[
  \inferrule[irefine-map]
    { Χ  = \many{σ_i ↦ [χ_{1i}, …, χ_{ki}] ⇒ [χ'_{1i}, …, χ'_{ki}]}{i < n} \\\\
      Χ' = \many{σ_i ↦ χ_{1i} ⇒ χ'_{1i}, …, σ_i ↦ χ_{ki} ⇒ χ'_{ki}}{i < n} \\\\
      Γ ⊢ α → β ▷ Χ' ⇝ f }
    { Γ ⊢ α\,\mlist → β\,\mlist ▷ Χ ⇝ λl{:}α\,\mlist.\,\mkwd{map}\,f\,l }
\]
Intuitively, this rule synthesizes a function $f$ for $\mkwd{map}$ by breaking up the input/output examples between lists point-wise, creating new examples that describe the required behavior of $f$.
Our $E$-guessing rules provide a necessary fall back for when these shortcuts are not possible.

\subsection{Forwards and Backwards Search}

We can interpret synthesis in \lsyn{} as proof search through the Curry-Howard isomorphism~\citep{william-curry-1980} where program synthesis at a particular type is equivalent to proof search of a particular proposition.
In this light, \lsyn{} specifies a \emph{backwards search} for a satisfying program.
That is, \lsyn{} exclusively refines the goal during the synthesis process.
Other proof search-inspired also take advantage of backwards search, \eg, \textsc{InSynth}~\citep{gvero-pldi-2013}.

In contrast, we could instead perform a \emph{forwards search} by refining our context during synthesis.
Consider extending our context so that it hold arbitrary expressions rather than just variables.
Now, starting with our free variables, we can incrementally grow the context by enumerating well-typed terms in some order, for example, increasing size.
At each step, we can check to see if we can construct a program from the \emph{components} in the context that satisfies our examples.
This style of forwards search-based program synthesis is performed by \textsc{Escher}~\citep{albarghouthi-cav-2013} which creates a satisfying program (in a Lisp-like programming language) from a set of components built up from atoms.

There are benefits and drawbacks to employing either a backwards or forwards-based program search.
Our backwards search allows us to decompose examples \emph{during} goal refinement which allows us to greatly cut down the space of programs during synthesis.
However, in the presence of other types such as products, pure backwards search can be inefficient because it introduces unnecessary non-determinism into the search process.
To obtain the best of both worlds, we can adopt a search strategy that combines backwards and forwards by using \emph{focusing}.
\citet{scherer-icfp-2015} use a proof search strategy equipped with focusing to efficiently find inhabitants of some given type, and we briefly explore the addition of focusing to \lsyn{} in conjunction with products in \autoref{subsec:tuple-efficiency}.

\subsection{Type-directed Synthesis Systems}

An alternative view of \lsyn{} is that it solves the \emph{type inhabitation problem}---given a type, produce an inhabitant of that type.
Several other systems have phrased the synthesis problem in terms of type inhabitation.
For example, \textsc{Prospector}~\citep{mandelin-pldi-2005} auto-completes API queries by building up chains of object method calls called \emph{jungloids}.
The discovery of these jungloids is type-directed: given a pair of types $(τ_1, τ_2)$ find a chain of jungloids whose composition has type $τ_1 → τ_2$.
It is analogous to repeated applications of our $E$-guessing rules, in particular \rulename{eguess-app} which computes one such function in this chain.

\citet{perelman-pldi-2012} use types to provide auto-completion for the C\# programming language.
To do this, they generate well-typed program completions for a given completion query and rank those programs according to a number of heuristics including depth, name, namespace, and \emph{type distance}---the distance between two types in the type hierarchy.
\textsc{InSynth}~\citep{gvero-pldi-2013} performs similar sorts of auto-completion for the Scala programming language, but it draws its rankings or weights of components by analyzing corpora of code.
These sorts of heuristics are also admissible in our system, similar to the example refinement heuristics discussed previously.
In particular, these rankings could be adapted to the $E$-term generation algorithms we discuss in \autoref{ch:implementation} to prioritize particular $E$-terms over others.

Finally, it is worthwhile to note that none of these type-directed systems that we have discussed include examples as part of their specification.
Really, \lsyn{} solves the \emph{type inhabitation and example satisfaction problem} where we are interested in finding programs of a particular type that satisfy the given examples.
The example-refining techniques that we have developed here could be adapted to these systems, so that they can benefit from the additional specification power that examples provide.
