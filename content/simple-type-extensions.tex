\lsyn{} is a core calculus for program synthesis with types.
As such, it only contains the bare essence of a typed, functional programming language, namely lambdas and function application.
However, one of the appeals of the type-directed synthesis approach is that we derive the synthesis judgment directly from the type checking judgment; we merely flip its inputs and outputs.
In other words, type checking immediately gives us insight into program synthesis for new types!

In this chapter, we explore the process of integrating new types into \lsyn{} by considering a number of additional basic types: products, sums, and records.
By playing this game with the type system of flipping inputs and outputs, we learn how to generate terms and refine examples of these types.
In some cases, this is sufficient to synthesize this type within \lsyn{}.
However in other cases, especially with more complex types that we consider in future chapters, we must do additional work to properly synthesize programs of those types.

\todo{Add note about Jonathan's and Rohan's contributions to products and records, respectively.}

\section{Products}

\begin{figure}
  \begin{center}
    \begin{tabular}{>{$}l<{$} >{$}r<{$} >{$}l<{$} l}
      τ  & \bnfdef & … \bnfalt τ_1 × τ_2 & Types \\
      e  & \bnfdef & … \bnfalt \mfst{e} \bnfalt \msnd{e} \bnfalt (e_1, e_2) & Terms \\
      v  & \bnfdef & … \bnfalt (v_1, v_2) & Values \\
      ℰ  & \bnfdef & … \bnfalt (ℰ, e) \bnfalt (v, ℰ) & Evaluation Contexts \\
      E  & \bnfdef & … \bnfalt \mfst{E} \bnfalt \msnd{E} & Elimination Terms \\
      I  & \bnfdef & … \bnfalt (I_1, I_2) & Introduction Terms \\
      χ  & \bnfdef & … \bnfalt (χ_1, χ_2) & Example Values \\
    \end{tabular} \\[12pt]
    \hrule
    \begin{gather*}
      \fbox{$Γ ⊢ e : τ$} \qquad
        \inferrule[t-pair]
          {Γ ⊢ e_1 : τ_1 \\ Γ ⊢ e_2 : τ_2}
          {Γ ⊢ (e_1, e_2) : τ_1 × τ_2} \qquad
        \inferrule[t-fst]
          {Γ ⊢ e : τ_1 × τ_2}
          {Γ ⊢ \mfst{e} : τ_1} \qquad
        \inferrule[t-snd]
          {Γ ⊢ e : τ_1 × τ_2}
          {Γ ⊢ \msnd{e} : τ_2} \\
      \fbox{$Γ ⊢ E ⇒ τ$} \qquad
        \inferrule[t-Efst]
          {Γ ⊢ E ⇒ τ_1 × τ_2}
          {Γ ⊢ \mfst{E} ⇒ τ_1} \qquad
        \inferrule[t-Esnd]
          {Γ ⊢ E ⇒ τ_1 × τ_2}
          {Γ ⊢ \msnd{E} ⇒ τ_2} \\
      \fbox{$Γ ⊢ I ⇐ τ$} \qquad
        \inferrule[t-Ipair]
          {Γ ⊢ I_1 ⇐ τ_1 \\ Γ ⊢ I_2 ⇐ τ_2}
          {Γ ⊢ (I_1, I_2) ⇐ τ_1 × τ_2} \\
      \fbox{$e ⟶ e'$} \qquad
        \inferrule[eval-fst]
          {}
          {\mfst{(v_1, v_2)} ⟶ v_1} \qquad
        \inferrule[eval-snd]
          {}
          {\msnd{(v_1, v_2)} ⟶ v_2} \\
      \fbox{$Γ ⊢ τ ⇝ E$} \qquad
        \inferrule[eguess-fst]
          {Γ ⊢ τ_1 × τ_2 ⇝ E}
          {Γ ⊢ τ_1 ⇝ \mfst{E}} \qquad
        \inferrule[eguess-snd]
          {Γ ⊢ τ_1 × τ_2 ⇝ E}
          {Γ ⊢ τ_2 ⇝ \msnd{E}} \\
      \fbox{$Γ ⊢ τ ▷ Χ ⇝ I$} \qquad
        \inferrule[irefine-prod]
          {Χ = \many{σ_i ↦ (v_{1i}, v_{2i})}{i < n} \\ \mfun{proj}(Χ) = (Χ_1, Χ_2) \\\\
           Γ ⊢ τ_1 ▷ Χ_1 ⇝ I_1 \\ Γ ⊢ τ_2 ▷ Χ_2 ⇝ I_2}
          {Γ ⊢ τ_1 × τ_2 ▷ Χ ⇝ (I_1, I_2)} \\
      \fbox{$Γ ⊢ χ : τ$} \qquad
        \inferrule[t-ex-pair]
          {Γ ⊢ χ_1 : τ_1 \\ Γ ⊢ χ_2 : τ_2}
          {Γ ⊢ (χ_1, χ_2) : τ_1 × τ_2} \qquad
      \fbox{$v ≃ χ$} \qquad
        \inferrule[eq-pair]
          {v_1 ≃ χ_1 \\ v_2 ≃ χ_2}
          {(v_1, v_2) ≃ (χ_1, χ_2)}
    \end{gather*}
    \[
      \begin{array}{l}
        \mfun{proj}(\many{σ_i ↦ (v_{1i}, v_{2i})}{i < n}) = (Χ_1, Χ_2) \\
        \quad\text{where}\;Χ_1 = \many{σ_i ↦ v_{1i}}{i < n}, Χ_2 = \many{σ_i ↦ v_{2i}}{i < n}
      \end{array}
    \]
  \end{center}
  \hrule
  \caption{\lsyn{} products}
  \label{fig:lsyn-products}
\end{figure}


\section{Records}

\begin{figure}
  \begin{center}
    \begin{tabular}{>{$}l<{$} >{$}r<{$} >{$}l<{$} l}
      τ  & \bnfdef & … \bnfalt \{\many{l_i{:}τ_{i}}{i < m}\} & Types \\
      e  & \bnfdef & … \bnfalt \{\many{l_i = e_i}{i < m}\} \bnfalt e.l & Terms \\
      v  & \bnfdef & … \bnfalt \{\many{l_i = v_i}{i < m}\} & Values \\
      ℰ  & \bnfdef & … \bnfalt \{l_1 = v_1, …, l = ℰ, …, l_m = e_m\} \bnfalt ℰ.l & Evaluation Contexts \\
      E  & \bnfdef & … \bnfalt E.l & Elimination Forms \\
      I  & \bnfdef & … \bnfalt \{\many{l_i = I_i}{i < m}\} & Introduction Forms \\
      χ  & \bnfdef & … \bnfalt \{\many{l_i = χ_i}{i < m}\} & Example Values \\
    \end{tabular} \\[12pt]
    \hrule
    \begin{gather*}
      \fbox{$Γ ⊢ e : τ$} \qquad
        \inferrule[t-record]
          {\many{Γ ⊢ e_i : τ_i}{i <m}}
          {Γ ⊢ \{\many{l_i = e_i}{i < m}\} : \{\many{l_i{:}τ_{i}}{i < m}\}} \qquad
        \inferrule[t-rproj]
          {Γ ⊢ e : \{\many{l_i{:}τ_{i}}{i < m}\}}
          {Γ ⊢ e.l_i : τ_i} \\
      \fbox{$Γ ⊢ I ⇐ τ$} \qquad
        \inferrule[t-Irecord]
          {\many{Γ ⊢ I_i ⇐ τ_i}{i <m}}
          {Γ ⊢ \{\many{l_i = I_i}{i < m}\} : \{\many{l_i{:}τ_{i}}{i < m}\}} \\
      \fbox{$Γ ⊢ χ : τ$} \qquad
        \inferrule[t-ex-record]
          {\many{Γ ⊢ χ_i : τ_i}{i <m}}
          {Γ ⊢ \{\many{l_i = χ_i}{i < m}\} : \{\many{l_i{:}τ_{i}}{i < m}\}} \\
      \fbox{$Γ ⊢ E ⇒ τ$} \qquad
        \inferrule[t-rproj]
          {Γ ⊢ e ⇒ \{\many{l_i{:}τ_{i}}{i < m}\}}
          {Γ ⊢ e.l_i ⇒ τ_i} \qquad
      \fbox{$e ⟶ e'$} \qquad
        \inferrule[eval-rproj]
          {}
          {\{\many{l_i = v_i}{i < m}\}.l_i ⟶ v_i} \\
      \fbox{$Γ ⊢ τ ⇝ E$} \qquad
        \inferrule[eguess-rproj]
          {Γ ⊢ \{\many{l_i{:}τ_i}{i < m}\} ⇝ E}
          {Γ ⊢ τ_i ⇝ E.l_i} \qquad
      \fbox{$v ≃ χ$} \qquad
        \inferrule[eq-record]
          {\many{v_i ≃ χ_i}{i < m}}
          {\many{l_i = v_i}{i < m} ≃ \many{l_i = χ_i}{i < m}} \\
      \fbox{$Γ ⊢ τ ▷ Χ ⇝ I$} \qquad
        \inferrule[irefine-record]
          {Χ = \many{σ_j ↦ \{l_1 = χ_{1j}, …, l_m = χ_{mj}\}}{j < n} \\\\
           \mfun{rproj}(Χ) = Χ_1, …, Χ_m \\ \many{Γ ⊢ τ_i ▷ Χ_i ⇝ I_i}{i < m}}
          {Γ ⊢ \{\many{l_i{:}τ_i}{i < m}\} ▷ Χ ⇝ \{\many{l_i = I_i}{i < m}\}}
    \end{gather*}
    \[
      \begin{array}{l}
        \mfun{rproj}(\many{σ_j ↦ \{l_1 = χ_{1j}, …, l_m = χ_{mj}\}}{j < n}) = Χ_1, …, Χ_m \\
        \quad \textrm{where} \\
        \qquad ∀i ∈ 1, …, m.\,Χ_i = \many{σ_j ↦ χ_{ij}}{i < m}
      \end{array}
    \]
  \end{center}
  \hrule
  \caption{\lsyn{} records definitions}
  \label{fig:lsyn-records-defn}
\end{figure}


\section{Sums}

\begin{figure}
  \begin{center}
    \begin{tabular}{>{$}l<{$} >{$}r<{$} >{$}l<{$} l}
      τ  & \bnfdef & … \bnfalt τ_1 + τ_2 & Types \\
      e  & \bnfdef & … \bnfalt \mmatch\;e\;\mwith\;\minl{x_1} → e_1 {\scriptstyle\bnfalt} \minr{x_2} → e_2 \bnfalt \minl{e} \bnfalt \minr{e} & Terms \\
      v  & \bnfdef & … \bnfalt \minl{v} \bnfalt \minr{v} & Values \\
      ℰ  & \bnfdef & … \bnfalt \mmatch\;ℰ\;\mwith\;\minl{x_1} → e_1 {\scriptstyle\bnfalt} \minr{x_2} → e_2 & Eval Ctx \\
      E  & \bnfdef & … & Elim \\
      I  & \bnfdef & … \bnfalt \mmatch\;E\;\mwith\;\minl{x_1} → I_1 {\scriptstyle\bnfalt} \minr{x_2} → I_2 \bnfalt \minl{I} \bnfalt \minr{I} & Intros \\
      χ  & \bnfdef & … \bnfalt \minl{χ} \bnfalt \minr{χ} & Ex. Values \\
    \end{tabular} \\[12pt]
    \hrule
    \begin{gather*}
      \fbox{$Γ ⊢ e : τ$} \qquad
        \inferrule[t-inl]
          {Γ ⊢ e : τ_1}
          {Γ ⊢ \minl{e} : τ_1 + τ_2} \qquad
        \inferrule[t-inr]
          {Γ ⊢ e : τ_2}
          {Γ ⊢ \minr{e} : τ_1 + τ_2} \\
        \inferrule[t-match]
          {Γ ⊢ e : τ_1 + τ_2 \\\\
           x_1{:}τ_1, Γ ⊢ e_1 : τ \\ x_2{:}τ_2, Γ ⊢ e_2 : τ}
          {Γ ⊢ \mmatch\;e\;\mwith\;\minl{x_1} → e_1 \bnfalt \minr{x_2} → e_2 : τ} \\
      \fbox{$Γ ⊢ I ⇐ τ$} \qquad
        \inferrule[t-Iinl]
          {Γ ⊢ I ⇐ τ_1}
          {Γ ⊢ \minl{I} ⇐ τ_1 + τ_2} \qquad
        \inferrule[t-Iinr]
          {Γ ⊢ I ⇐ τ_2}
          {Γ ⊢ \minr{I} ⇐ τ_1 + τ_2} \\
        \inferrule[t-Imatch]
          {Γ ⊢ E ⇒ τ_1 + τ_2 \\\\
           x_1{:}τ_1, Γ ⊢ I_1 ⇐ τ \\ x_2{:}τ_2, Γ ⊢ I_2 ⇐ τ}
          {Γ ⊢ \mmatch\;E\;\mwith\;\minl{x_1} → I_1 \bnfalt \minr{x_2} → I_2 : τ} \\
      \fbox{$Γ ⊢ χ : τ$} \qquad
        \inferrule[t-ex-inl]
          {Γ ⊢ χ : τ_1}
          {Γ ⊢ \minl{χ} : τ_1 + τ_2} \qquad
        \inferrule[t-ex-inr]
          {Γ ⊢ χ : τ_2}
          {Γ ⊢ \minr{χ} ⇐ τ_1 + τ_2} \\
      \fbox{$e ⟶ e'$} \qquad
        \inferrule[eval-match-inl]
          {}
          {\mmatch\;\minl{v}\;\mwith\;\minl{x_1} → e_1 \bnfalt \minr{x_2} → e_2 ⟶ [v/x_1]e_1} \\
        \inferrule[eval-match-inr]
          {}
          {\mmatch\;\minr{v}\;\mwith\;\minl{x_1} → e_1 \bnfalt \minr{x_2} → e_2 ⟶ [v/x_2]e_2}
    \end{gather*}
  \end{center}
  \hrule
  \caption{\lsyn{} sums definitions}
  \label{fig:lsyn-sums-defn}
\end{figure}

\begin{figure}
  \begin{center}
    \begin{gather*}
      \fbox{$Γ ⊢ τ ▷ Χ ⇝ I$} \qquad
        \inferrule[irefine-sum-inl]
          {Γ ⊢ τ_1 ▷ \many{σ_i ↦ χ_i}{i < n} ⇝ I}
          {Γ ⊢ τ_1 + τ_2 ▷ \many{σ_i ↦ \minl[τ_1 + τ_2]{χ_i}}{i < n} ⇝ \minl[τ_1 + τ_2]{I}} \\
        \inferrule[irefine-sum-inr]
          {Γ ⊢ τ_2 ▷ \many{σ_i ↦ χ_i}{i < n} ⇝ I}
          {Γ ⊢ τ_1 + τ_2 ▷ \many{σ_i ↦ \minr[τ_1 + τ_2]{χ_i}}{i < n} ⇝ \minl[τ_1 + τ_2]{I}} \\
        \inferrule[irefine-match]
          {Γ ⊢ τ_1 + τ_2 ⇝ E \\\\
           \mfun{distribute}(E, Χ) = (Χ_l, Χ_r) \\\\
           x_1{:}τ_1, Γ ⊢ τ ▷ Χ_l ⇝ I_1 \\ x_2{:}τ_2, Γ ⊢ τ ▷ Χ_r ⇝ I_2}
          {Γ ⊢ τ ▷ Χ ⇝ \mmatch\;E\;\mwith\;\minl{x_1} → I_1 \semisep \minr{x_2} → I_2} \\
      \fbox{$v ≃ χ$} \qquad
        \inferrule[eq-inl]
          {v ≃ χ}
          {\minl[τ_1 + τ_2]{v} ≃ \minl[τ_1 + τ_2]{χ}} \qquad
        \inferrule[eq-inr]
          {v ≃ χ}
          {\minr[τ_1 + τ_2]{v} ≃ \minr[τ_1 + τ_2]{χ}}
    \end{gather*}
    \[
      \begin{array}{l}
        \mfun{distribute}(E, Χ) = (X_l, X_r) \\
        \quad \textrm{where} \\
        \qquad X_l = \{[v/x_1]σ ↦ χ \bnfalt σ ↦ χ ∈ Χ ∧ σ(E) ⟶^* \minl[τ_1 + τ_2]{v}\} \\
        \qquad X_r = \{[v/x_2]σ ↦ χ \bnfalt σ ↦ χ ∈ Χ ∧ σ(E) ⟶^* \minr[τ_1 + τ_2]{v}\}
      \end{array}
    \]
  \end{center}
  \hrule
  \caption{\lsyn{} sums: synthesis rules}
  \label{fig:lsyn-sums-synthesis}
\end{figure}


\section{Let Binding}
