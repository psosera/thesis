In this thesis, we have explored the integration of type theory into program synthesis for typed, functional programming languages.
By using type theory as the basis of our synthesis techniques, we were able to:
\begin{enumerate}
  \item Build core calculi for program synthesis rooted in the simply-typed lambda calculus and its extensions.
  \item Exploit the logical nature of types to greatly reduce the search space of possible programs, refine specification, and decompose synthesis problems into smaller, independent synthesis sub-problems.
  \item Gain insight into how to synthesis programs with advanced languages features like recursion and polymorphism by inspection of the type system.
  \item Leverage the power of the proof search to help optimize our search procedures in a variety of ways.
  \item Reason carefully about the behavior of our core program synthesis calculi, proving soundness and completeness of the relevant synthesis procedure when possible and explaining why these properties fail when they do not hold.
\end{enumerate}
In short, we have laid down a foundation for program synthesis with types and demonstrated its effectiveness.
We hope that others can build upon our work to integrate types into other existing synthesis systems or begin exploring the space of program synthesis in the presence of rich types.

\section{Future Directions}
\label{sec:future-directions}

Nevertheless, there are many areas left to explore to increase the expressiveness of the program synthesis calculi we have developed, improve the performance of \myth{}, or apply these program synthesis techniques to solve problems in more targetted domains.
We close by exploring these future directions in more detail.

\subsection{Synthesis with Rich Types}
\label{subsec:synthesis-with-rich-types}

\subsection{Additional Specification}
\label{subsec:additional-specification}

\subsection{Enumeration Modulo Equivalences}
\label{subsec:enumeration-modulo-equivalences}

\subsection{Applications of Program Synthesis with Types}
\label{subsec:applications-of-program-synthesis-with-types}
