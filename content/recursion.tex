So far, we have considered adding a variety of basic types to \lsyn{}.
While these types allow us to synthesize programs that more closely match those found in actual functional programming languages, they have not significantly changed the expressiveness of our core language.
Next we will consider adding recursion to \lsyn{} which greatly increases its expressiveness.

\subsection{\texorpdfstring{$μ$}{μ} Types}

\begin{figure}
  \begin{center}
    \begin{tabular}{>{$}l<{$} >{$}r<{$} >{$}l<{$} l}
      τ  & \bnfdef & … \bnfalt α \bnfalt μα.\,τ & Types \\
      e  & \bnfdef & … \bnfalt \mfold{e} \bnfalt \munfold{e} & Terms \\
      v  & \bnfdef & … \bnfalt \mfold{v} & Values \\
      ℰ  & \bnfdef & … \bnfalt \mfold{ℰ} \bnfalt \munfold{ℰ} & Evaluation Contexts \\
      E  & \bnfdef & … \bnfalt \munfold{E} & Elimination Terms \\
      I  & \bnfdef & … \bnfalt \mfold{I} & Introduction Terms \\
      χ  & \bnfdef & … \bnfalt \mfold{χ} & Example Values \\
    \end{tabular} \\[12pt]
    \hrule
    \begin{gather*}
      \fbox{$Γ ⊢ e : τ$} \qquad
        \inferrule[t-fold]
          {Γ ⊢ e : [μα.\,τ/α] τ}
          {Γ ⊢ \mfold{e} : μα.\,τ} \qquad
        \inferrule[t-unfold]
          {Γ ⊢ e : μα.\,τ}
          {Γ ⊢ \munfold{e} : [μα.\,τ/α] τ} \\
      \fbox{$Γ ⊢ E ⇒ τ$} \qquad
        \inferrule[t-unfold]
          {Γ ⊢ E ⇒ μα.\,τ}
          {Γ ⊢ \munfold{E} ⇒ [μα.\,τ/α] τ} \\
      \fbox{$Γ ⊢ I ⇐ τ$} \qquad
        \inferrule[t-Ifold]
          {Γ ⊢ I ⇐ [μα.\,τ] τ}
          {Γ ⊢ \mfold{I} ⇐ μα.\,τ} \\
      \fbox{$e ⟶ e'$} \qquad
        \inferrule[eval-unfold-fold]
          {}
          {\munfold{\mfold{v}} ⟶ v} \\
      \fbox{$Γ ⊢ τ ⇝ E$} \qquad
        \inferrule[eguess-unfold]
          {Γ ⊢ μα.\,τ ⇒ E}
          {Γ ⊢ [μα.\,τ/α] τ ⇒ \munfold{E}} \\
      \fbox{$Γ ⊢ τ ▷ Χ ⇝ I$} \qquad
        \inferrule[irefine-mu]
          {Χ = \many{σ_i ↦ \mfold{χ_i}}{i < n} \\ Γ ⊢ [μα.\,τ/α] τ ▷ \mfun{unfold}(Χ') ⇝ I}
          {Γ ⊢ μα.\,τ ▷ Χ ⇝ \mfold{I}} \\
      \fbox{$Γ ⊢ χ : τ$} \qquad
        \inferrule[t-ex-fold]
          {Γ ⊢ χ : [μα.\,τ/α] τ}
          {Γ ⊢ \mfold{χ} : μα.\,τ} \qquad
      \fbox{$v ≃ χ$} \qquad
        \inferrule[eq-pair]
          {v ≃ χ}
          {\mfold{v} ≃ \mfold{χ}}
    \end{gather*}
    \[
      \begin{array}{l}
        \mfun{unfold}(\many{σ_i ↦ \mfold{χ_i}}{i < n}) = \many{σ_i ↦ χ_i}{i < n} \\
      \end{array}
    \]
  \end{center}
  \hrule
  \caption{\lsyn{} $μ$ types}
  \label{fig:lsyn-mu-defn}
\end{figure}


One way to express recursion within a typed lambda calculus is with $μ$-types.
The type $μα.\,τ$ binds a recursive occurrence of a type to the type variable $α$ which appears in its definition $τ$.

\begin{figure}
  \begin{center}
    \begin{tabular}{>{$}l<{$} >{$}r<{$} >{$}l<{$}}
      τ  & \bnfdef & T \bnfalt τ_1 → τ_2 \\
      p  & \bnfdef & C(x_1, …, x_k) \\
      pf & \bnfdef & \many{v_i ⇒ χ_i}{i < m} \\
      e  & \bnfdef & x \bnfalt e_1\;e_2 \bnfalt pf \bnfalt \mkwd{fix}\;f\;(x{:}τ_1) : τ_2 = e \bnfalt C\;(e_1, …, e_k) \\
         & \bnfalt & \mmatch\;e\;\mwith\;\many{p_i → e_i}{i < m} \bnfalt \mNoMatch \\
      v  & \bnfdef & pf \bnfalt \mfix\;f\;(x{:}τ_1) : τ_2 = e \bnfalt C\;(v_1, …, v_k) \bnfalt \mNoMatch \\
      ℰ  & \bnfdef & ◻ \bnfalt ℰ\;e \bnfalt v\;ℰ \bnfalt C\;(v_1, …, ℰ, …, e_k) \\
         & \bnfalt & \mmatch\;ℰ\;\mwith\;\many{p_i → e_i}{i < m} \\
      b  & \bnfdef & · \bnfalt \mrec \bnfalt \marg{f} \bnfalt \mdec{f} \\
      Γ  & \bnfdef & · \bnfalt x{:}τ\{b\}, Γ \\
      Σ  & \bnfdef & · \bnfalt C{:}τ_1 * … * τ_k → T, Σ \\
      \\
      E  & \bnfdef & x \bnfalt E\;I \\
      I  & \bnfdef & E \bnfalt \mkwd{fix}\;f\;(x{:}τ_1) : τ_2 = I \\
         & \bnfalt & C(I_1, …, I_k) \bnfalt \mmatch\;E\;\mwith\;\many{p_i → I_i}{i < m} \\
      σ  & \bnfdef & · \bnfalt [v/x]σ \\
      χ  & \bnfdef & pf \bnfalt C(χ_1, …, χ_k) \\
      Χ  & \bnfdef & · \bnfalt σ ↦ χ, Χ \\
    \end{tabular}
  \end{center}
  \hrule
  \caption{\mlsyn{} syntax}
  \label{fig:mlsyn-defn}
\end{figure}


\begin{figure}
  \begin{center}
    \begin{gather*}
      \fbox{$Σ;Γ ⊢ e : τ$} \qquad
        \inferrule[t-var]
          {x{:}τ ∈ Γ}
          {Σ;Γ ⊢ x : τ} \qquad
        \inferrule[t-unit]
          { }
          {Σ;Γ ⊢ () : \munit} \\
        \inferrule[t-app]
          {Σ;Γ ⊢ e_1 : τ_1 → τ_2 \\
           Σ;Γ ⊢ e_2 : τ_1 \\\\
           \mfun{struct}(Γ, e_1, e_2)}
          {Σ;Γ ⊢ e_1\;e_2 : τ_2} \qquad
        \inferrule[t-ctor]
          {C : τ_1, …, τ_k → T ∈ Σ \\ \many{Σ;Γ ⊢ e_i : τ_i}{i < k}}
          {Σ;Γ ⊢ C\;(e_1, …, e_k) : T} \\
        \inferrule[t-pf]
          {\many{Σ;Γ ⊢ v_i : τ_1}{i < m} \\ \many{Σ;Γ ⊢ χ_i : τ_2}{i < m}}
          {Σ;Γ ⊢ \many{v_i ⇒ χ_i}{i < m} : τ_1 → τ_2} \qquad
        \inferrule[t-fix]
          {Σ;f{:}τ_1 → τ_2 \{\mrec\}, x{:}τ_1 \{\marg{f}\}, Γ ⊢ e : τ_2}
          {Σ;Γ ⊢ \mfix\;f\;(x{:}τ_1) : τ_2 = e : τ_1 → τ_2} \\
        \inferrule[t-match]
          {Σ;Γ ⊢ e : T \\ \mfun{complete}(Σ, \many{p_i}{i < m}, T) \\\\
          \many{\mfun{binders}(Γ, e, p_i) = Γ_i}{i < m} \\ \many{Σ; Γ_i, Γ ⊢ e_i : τ}{i < m}}
          {Σ;Γ ⊢ \mmatch\;e\;\mwith\;\many{p_i → e_i}{i < m} : τ} \\
      \fbox{$\mfun{struct}(e_1, e_2)$} \qquad
        \inferrule[struct-var-rec]
          {f{:}τ_1 → τ_2\{\mrec\} ∈ Γ \\
           x{:}τ_1\{\mdec{f}\} ∈ Γ}
          {\mfun{struct}(Γ, f, x)} \\
        \inferrule[struct-var-not-rec]
          {f{:}τ_1 → τ_2\{b\} ∈ Γ \\ b ≠ \mrec}
          {\mfun{struct}(Γ, f, e)} \qquad
        \inferrule[struct-not-var]
          {e_1 ≠ f}
          {\mfun{struct}(e_1, e_2)} \\
      \fbox{$\mfun{complete}(Σ, \many{p_i}{i < m}, T)$} \qquad
        \inferrule[complete]
          {C ∈ τ_1 * … * τ_k → T ∈ Σ ↔ C(x_1, …, x_k) \in \many{p_i}{i < m}}
          {\mfun{complete}(Σ, \many{p_i}{i < m})}
    \end{gather*}
    \[
      \begin{array}{l}
        \mfun{binders}(Γ, e, C(x_1, …, x_k)) = x_1{:}τ_1 \{b_1\}, …, x_k{:}τ_k \{b_k\} \\
        \quad \textrm{where} \\
        \qquad ∀i ∈ 1, …, k.\, b_i = \begin{cases}
          \mdec{f} & e = x, x{:}τ \{b\} ∈ Γ, b = \marg{f} ∨ b = \mrec{f}, τ_i = τ \\
            · & \textrm{otherwise}
          \end{cases}
      \end{array}
    \]
  \end{center}
  \hrule
  \caption{\mlsyn{} external language type checking}
  \label{fig:mlsyn-ext-types}
\end{figure}


\begin{figure}
  \begin{center}
    \begin{gather*}
      \fbox{$Σ;Γ ⊢ E ⇒ τ$} \qquad
        \inferrule[t-Evar]
          {x{:}τ ∈ Γ}
          {Σ;Γ ⊢ x ⇐ τ} \qquad
        \inferrule[t-Eapp]
          {Σ;Γ ⊢ E ⇒ τ_1 → τ_2 \\
           Σ;Γ ⊢ I ⇐ τ_1 \\\\
           \mfun{struct}(Γ, E, I)}
          {Σ;Γ ⊢ E\;I : τ_2} \\
      \fbox{$Σ;Γ ⊢ I ⇐ τ$} \qquad
        \inferrule[t-Ielim]
          {Σ;Γ ⊢ E ⇒ τ}
          {Σ;Γ ⊢ I ⇐ τ} \qquad
        \inferrule[t-Ictor]
          {C : τ_1, …, τ_k → T ∈ Σ \\ \many{Σ;Γ ⊢ I_i ⇐ τ_i}{i < k}}
          {Σ;Γ ⊢ C\;(I_1, …, I_k) ⇐ T} \\
        \inferrule[t-fix]
          {Σ;f{:}τ_1 → τ_2 \{\mrec\}, x{:}τ_1 \{\marg{f}\}, Γ ⊢ I ⇐ τ_2}
          {Σ;Γ ⊢ \mfix\;f\;(x{:}τ_1) : τ_2 = I ⇐ τ_1 → τ_2} \\
        \inferrule[t-match]
          {Σ;Γ ⊢ E ⇒ T \\ \mfun{complete}(Σ, \many{p_i}{i < m}, T) \\\\
          \many{\mfun{binders}(Γ, E, p_i) = Γ_i}{i < m} \\ \many{Σ; Γ_i, Γ ⊢ I_i ⇐ τ}{i < m}}
          {Σ;Γ ⊢ \mmatch\;E\;\mwith\;\many{p_i → e_i}{i < m} ⇐ τ} \\
      \fbox{$\mfun{struct}(E, I)$} \qquad
        \inferrule[struct-var-rec]
          {f{:}τ_1 → τ_2\{\mrec\} ∈ Γ \\
           x{:}τ_1\{\mdec{f}\} ∈ Γ}
          {\mfun{struct}(Γ, f, x)} \\
        \inferrule[struct-var-not-rec]
          {f{:}τ_1 → τ_2\{b\} ∈ Γ \\ b ≠ \mrec}
          {\mfun{struct}(Γ, f, I)} \qquad
        \inferrule[struct-not-var]
          {E ≠ f}
          {\mfun{struct}(E, I)} \\
      \fbox{$\mfun{complete}(Σ, \many{p_i}{i < m}, T)$} \qquad
        \inferrule[complete]
          {C ∈ τ_1 * … * τ_k → T ∈ Σ ↔ C(x_1, …, x_k) \in \many{p_i}{i < m}}
          {\mfun{complete}(Σ, \many{p_i}{i < m})}
    \end{gather*}
    \[
      \begin{array}{l}
        \mfun{binders}(Γ, E, C(x_1, …, x_k)) = x_1{:}τ_1 \{b_1\}, …, x_k{:}τ_k \{b_k\} \\
        \quad \textrm{where} \\
        \qquad ∀i ∈ 1, …, k.\, b_i = \begin{cases}
          \mdec{f} & E = x, x{:}τ \{b\} ∈ Γ, b = \marg{f} ∨ b = \mrec{f}, τ_i = τ \\
            · & \textrm{otherwise}
          \end{cases}
      \end{array}
    \]
  \end{center}
  \hrule
  \caption{\mlsyn{} internal language type checking}
  \label{fig:mlsyn-int-types}
\end{figure}


\begin{figure}
  \begin{center}
    \begin{gather*}
      \fbox{$e → e'$} \qquad
        \inferrule[eval-ctx]
          {e ⟶ e'}
          {ℰ[e] ⟶ ℰ[e']} \qquad
        \inferrule[eval-nomatch]
          {}
          {ℰ[\mNoMatch] ⟶ \mNoMatch} \qquad
        \inferrule[eval-app]
          {}
          {(λx{:}τ.\,e)\;v ⟶ [v/x]e} \\
        \inferrule[eval-pf-good]
          {v ≃ v_j \\ j ∈ 1, …, m}
          {(\many{v_i ⇒ χ_i}{i < m})\;v → χ_j} \qquad
        \inferrule[eval-pf-bad]
          {∀i ∈ 1, …, m.\,v ≄ v_i}
          {(\many{v_i ⇒ χ_i}{i < m})\;v → \mNoMatch}
    \end{gather*}
    \[
      \begin{array}{l}
        \rulename{eval-match} \\
        \quad \mmatch\;C(v_1, …, v_k)\;\mwith\;p_1 → e_1 \bnfalt … \bnfalt C(x_1, …, x_k) → e \bnfalt … \bnfalt p_m → e_m \\
        \qquad ⟶ [v_1/x_1]…[v_k/x_k](e)
      \end{array}
    \]
    \begin{gather*}
      \fbox{$v ≃ v'$} \qquad
        \inferrule[eq-refl]
          {}
          {v ≃ v} \qquad
        \inferrule[eq-sym]
          {v' ≃ v}
          {v ≃ v'} \\
        \inferrule[eq-pf-pf]
          {∀i ∈ 1, …, m.\,∃j ∈ 1, …, n.\,v_i ≃ v_j ∧ χ_i ≃ χ_j \\\\
           ∀j ∈ 1, …, n.\,∃i ∈ 1, …, m.\,v_i ≃ v_j ∧ χ_i ≃ χ_j}
          {\many{v_i ⇒ χ_i}{i < m} ≃ \many{v_j ⇒ χ_j}{j < n}} \\
        \inferrule[eq-fix-px]
          {∀i ∈ 1, …, m.\,(\mfix\;f\;(x{:}τ_1) : τ_2 = e)\;v_i ⟶^* v ∧ v ≃ χ_i}
          {\mfix\;f\;(x{:}τ_1) : τ_2 = e ≃ \many{v_i ⇒ χ_i}{i < m}} \\
        \inferrule[eq-ctor]
          {v_{11} ≃ v_{21} \;…\; v_{1k} ≃ v_{2k}}
          {C(v_{11}, …, v_{1k}) ≃ C(v_{21}, …, v_{2k})}
    \end{gather*}
  \end{center}
  \hrule
  \caption{\mlsyn{} evaluation and compatibility rules}
  \label{fig:mlsyn-eval}
\end{figure}


\begin{figure}
  \begin{center}
    \begin{gather*}
      \fbox{$Σ;Γ ⊢ τ ⇝ E$} \qquad
        \inferrule[eguess-var]
          {x{:}τ ∈ Γ}
          {Σ;Γ ⊢ τ ⇝ x} \qquad
        \inferrule[eguess-app]
          {Σ;Γ ⊢ τ_1 → τ_2 ⇝ E \\
           Σ;Γ ⊢ τ_1 ▷ · ⇝ I \\\\
           \mfun{struct}(Γ, E, I)}
          {Σ;Γ ⊢ τ_2 ⇝ E\;I} \\
      \fbox{$I ⊨ Χ$} \qquad
        \inferrule[satisfies]
        {∀σ ↦ χ ∈ Χ.\, σ(I) ⟶^* v ∧ v ≃ χ}
        {I ⊨ Χ} \\
      \fbox{$Σ;Γ ⊢ τ ▷ Χ ⇝ I$} \qquad
        \inferrule[irefine-guess]
          {Σ;Γ ⊢ τ ⇝ E \\ E ⊨ Χ}
          {Σ;Γ ⊢ τ ▷ Χ ⇝ E} \qquad
        \inferrule[irefine-unit]
          {Χ = \many{σ_i ↦ ()}{i < n}}
          {Σ;Γ ⊢ \munit ▷ Χ ⇝ ()} \\
        \inferrule[irefine-base]
        {C : τ_1, …, τ_k → T ∈ Σ \\ X = \many{σ_i ↦ C(I_{i1}, …, I_{ik})}{i < n} \\\\
         \mfun{proj}(X) = C_1, …, C_k \\ \many{Σ;Γ ⊢ I_i ⇐ τ_i}{i < k}}
          {Σ;Γ ⊢ T ▷ Χ ⇝ C\;(I_1, …, I_k) ⇐ T} \\
        \inferrule[irefine-arr]
          {X = σ_1 ↦ pf_1, …, σ_n ↦ pf_n \\\\
           X' = \mfun{apply}(f, x, σ_1 ↦ pf_1) \concat … \concat \mfun{apply}(f, x, σ_n ↦ pf_n) \\\\
           Σ;f{:}τ_1 → τ_2 \{\mrec\}, x{:}τ_1 \{\marg{f}\}, Γ ⊢ I ▷ X' ⇝ τ_2}
          {Σ;Γ ⊢ τ_1 → τ_2 ▷ Χ ⇝ \mfix\;f\;(x{:}τ_1) : τ_2 = I} \\
        \inferrule[t-match]
          {Σ;Γ ⊢ T ⇝ E \\ \mfun{distribute}(Σ, T, X, E) = \many{(p_i, X'_i)}{i < m} \\\\
          \many{\mfun{binders}(Γ, E, p_i) = Γ_i}{i < m} \\ \many{Σ; Γ_i, Γ ⊢ τ ▷ Χ ⇝ I_i}{i < m}}
          {Σ;Γ ⊢ τ ▷ Χ ⇝ \mmatch\;E\;\mwith\;\many{p_i → e_i}{i < m}}
    \end{gather*}
  \end{center}
  \hrule
  \caption{\mlsyn{} synthesis}
  \label{fig:mlsyn-synthesis}
\end{figure}


\begin{figure}
  \begin{center}
    \[
      \begin{array}{l}
      \mfun{proj}(Χ) = \many{σ_i ↦ χ_{1i}}{i < n}, …, \many{σ_i ↦ χ_{ki}}{i < n} \\
      \quad \mathrm{where}\;Χ = \many{σ_i ↦ C(χ_{1i}, …, χ_{ki})}{i < n} \\[6pt]
      \mfun{apply}(f, x, σ ↦ pf) = \many{[pf/f][v_i/x] σ ↦ χ_i}{i < m} \\
      \quad \mathrm{where}\;pf = \many{v_i ⇒ χ_i}{i < m} \\[6pt]
      \mfun{distribute}(Σ, T, Χ, E) = (p_1, Χ'_1), …, (p_m, Χ'_m) \\
      \quad \mathrm{where} \\
      \qquad \mfun{ctors}(Σ, T) = C_1, …, C_m \\
      \qquad ∀i ∈ 1, …, n.\,p_i = \mfun{pattern}(Σ, C_i) \\
      \qquad ∀i ∈ 1, …, n.\,Χ'_i = [σ'σ ↦ χ \bnfalt σ ↦ χ ∈ Χ, σ(E) ⟶^* C_i(χ1, …, χ_k), \\
      \qquad\quad \mfun{vbinders}(p_i, χ_1, …, χ_k) = σ'] \\[6pt]
      \mfun{ctors}(Σ, T) = C_1, …, C_n \\
      \quad \mathrm{where}\;∀i ∈ 1, …, n.\,C_i : τ_1 * … * τ_k → T ∈ Σ \\[6pt]
      \mfun{pattern}(Σ, C) = C(x_1, …, x_k) \\
      \quad \mathrm{where}\;C : τ_1 * … * τ_k → T ∈ Σ \\[6pt]
      \mfun{vbinders}(p, e_1, …, e_k) = [e_1/x_1]…[e_k/x_k] \\
      \quad \mathrm{where}\;p = C(x_1, …, x_k)
      \end{array}
    \]
  \end{center}
  \hrule
  \caption{\mlsyn{} synthesis auxiliary functions}
  \label{fig:mlsyn-aux}
\end{figure}

